% Define glossary entries with \newglossaryentry{<label>}{<settings>}. Don't add a full stop at the end of the description as the compiler will do so automatically
\newglossaryentry{pileup}{
        name=pileup,
        description={The term ascribed to additional proton-proton collisions during a bunch crossing. Pileup interactions typically produce a large number of low-momentum particles}
}

\newglossaryentry{luminosity}{
        name=luminosity,
        description={Instantaneous luminosity is a measure of the collision rate in a particle accelerator (given in units of area per unit time). The integrated luminosity is the instantaneous luminosity integrated over time and is a metric for measuring the total amount of data delivered by an accelerator or collected by a detector}
}

\newglossaryentry{jet}{
        name=jet,
        description={A collimated shower of hadronic particles. High momentum quarks and gluons fragment and radiate; the resulting particles combine into hadrons due to colour confinement and deposit energy in the detector very close to each other and is reconstructed as a single physics object called a jet}
}

\newglossaryentry{svj}{
        name=semi-visible jet,
        description={A shower of standard model and dark hadrons from the decay of a leptophobic \PZprime or $\Phi$ mediator that couples the hidden sector to the standard model}
}

\newglossaryentry{bjet}{
        name=\Pqb-jet,
        description={A jet identified by a given algorithm or classifier as originating from a \Pqb quark}
}

\newglossaryentry{particleflow}{
        name=\textsc{Particle Flow} algorithm,
        description={An event reconstruction algorithm used in CMS. Information from all subdetectors is combined so that all stable particles can be identified. Jets are clustered, and complex objects such as \Pqb-jets are classified}
}

\newglossaryentry{antikt}{
        name=anti-\kt algorithm,
        description={A sequential clustering algorithm designed to group hadronic particles into jets with a radius parameter $R$. The transverse momentum is symbolised as \kt instead of \pt. In CMS, standard radius parameters are $R = \text{0.4}$ and $R = \text{0.8}$, referred to as AK4 and AK8 jets, respectively},
        symbol={anti-\kt}
}

% Maybe define it predominantly as missing transverse momentum instead. If so, use correct symbol
\newglossaryentry{met}{
        name=missing transverse energy,
        description={The negative vector sum of the transverse momentum of all particles in a collider event. It is sometimes abbreviated to ``MET'' with the symbol $\VEtmiss$, and also referred to in literature as ``missing transverse momentum'' ($\ptvecmiss$)},
        symbol={$\VEtmiss$}
}

\newglossaryentry{CR}{
        name=control region,
        description={A region of phase space orthogonal to the signal region, typically by requiring an object that would otherwise be vetoed in the signal region. A control region enriched in a background process present in the signal region (such as \wtolnupjets) can be used to model its influence more accurately},
}

\newglossaryentry{SB}{
        name=sideband,
        description={A region of phase space orthogonal to the signal region, typically by inverting kinematic requirements. Similar to a control region, a sideband enriched in a background present in the signal region (such as QCD) can be used to model it more accurately},
}

% Define acronyms with \newacronym{<label>}{<abbrv>}{<full>}
% Organisations
\newacronym{cern}{CERN}{\emph{Organisation Europ\'{e}enne pour la Recherche Nucl\'{e}aire} (European Organisation for Nuclear Research)}
\newacronym{fnal}{FNAL}{Fermi National Accelerator Laboratory}

% Colliders/accelerators
\newacronym{lhc}{LHC}{Large Hadron Collider}
\newacronym{hllhc}{HL-LHC}{High Luminosity Large Hadron Collider}
\newacronym{lep}{LEP}{Large Electron-Positron Collider}
\newacronym{ilc}{ILC}{International Linear Collider}
\newacronym{clic}{CLIC}{Compact Linear Collider}
\newacronym{fcc}{FCC}{Future Circular Collider}
\newacronym{psb}{PSB}{Proton Synchrotron Booster}
\newacronym{ps}{PS}{Proton Synchrotron}
\newacronym{sps}{SPS}{Super Proton Synchrotron}
\newacronym{isolde}{ISOLDE}{Isotope Separator On Line DEvice}

% Experiments
\newacronym{cms}{CMS}{Compact Muon Solenoid}
\newacronym{atlas}{ATLAS}{A Toroidal LHC ApparatuS}
\newacronym{alice}{ALICE}{A Large Ion Collider Experiment}
\newacronym{totem}{TOTEM}{TOTal Elastic and diffractive cross section Measurement}
\newacronym{moedal}{MoEDAL}{Monopole and Exotics Detector At the LHC}
\newacronym{faser}{FASER}{ForwArd Search ExpeRiment}
\newacronym{aegis}{AEGIS}{Antihydrogen Experiment: Gravity, Interferometry, Spectroscopy}
\newacronym{lz}{LZ}{LUX-ZEPLIN}

% CMS components/subdetectors
\newacronym{ecal}{ECAL}{electromagnetic calorimeter}
\newacronym{hcal}{HCAL}{hadron calorimeter}
\newacronym{hf}{HF}{hadron forward calorimeter}
\newacronym{l1}{L1}{Level-1}
\newacronym{l1t}{L1T}{Level-1 Trigger}
\newacronym{hlt}{HLT}{High-Level Trigger}
\newacronym{daq}{DAQ}{data acquisition}

% Theory
\newacronym{sm}{SM}{standard model}
\newacronym{bsm}{BSM}{beyond the standard model}
\newacronym{gut}{GUT}{Grand Unified Theory}
\newacronym{qcd}{QCD}{Quantum Chromodynamics}
\newacronym{qed}{QED}{Quantum Electrodynamics}
\newacronym{qft}{QFT}{quantum field theory}
\newacronym{lo}{LO}{leading order}
\newacronym{nlo}{NLO}{next-to-leading order}
\newacronym{nnlo}{NNLO}{next-to-next-to-leading order}

% Dark matter/BSM terms
\newacronym{dm}{DM}{dark matter}
\newacronym{mond}{MOND}{modified Newtonian dynamics}
\newacronym{wimp}{WIMP}{Weakly Interacting Massive Particle}
\newacronym{susy}{SUSY}{supersymmetry}
\newacronym{lsp}{LSP}{lightest supersymmetric particle}

% Misc.
\newacronym{bdt}{BDT}{boosted decision tree}
\newacronym{mc}{MC}{Monte Carlo}
\newacronym{vbf}{VBF}{vector boson fusion}
\newacronym{pu}{PU}{pileup}
\newacronym{pf}{PF}{\textsc{Particle Flow}}
\newacronym{cmssw}{\textsc{cmssw}}{CMS SoftWare}
\newacronym{gsf}{GSF}{gaussian sum filter}
\newacronym{aod}{AOD}{Analysis Object Data}
\newacronym{pog}{POG}{Physics Object Group}
\newacronym{jec}{JEC}{jet energy corrections}
\newacronym{jer}{JER}{jet energy resolution}
\newacronym{jes}{JES}{jet energy scale}
\newacronym{ev}{\eV}{electron volt}

% To reference glossary terms and acronyms in the document, see https://en.wikibooks.org/wiki/LaTeX/Glossary#Using_defined_terms for commands
% The general glossary commands \gls{}, \Gls{}, etc. can be used for acronyms as well. For example, if I want the first letter of "jet energy corrections" capitalised, even though it's an acronym I can do \Gls{jec}.
