\chapter{Conclusions}
\label{chap:conclusions}

\epigraph{Satisfaction lies in the effort, not in the attainment.}{--- Mahatma Gandhi}

\initial{S}cientific research can often be imagined as searching for bright city lights at the end of a dimly-lit, winding road. One hopes to reach it quickly with minimal setbacks. Despite somewhat stumbling in the dark and following the faintest strands, and unexpected twists and turns, progress is made through effort and determination even if the destination seems still out of sight. Reflecting on the work carried out over the course of this PhD and documented in this thesis, the seemingly unscalable wall that is the nature of dark matter has been modestly chipped away.


%=========================================================


\section{\texorpdfstring{\higgstoinv}{Higgs to invisible}}
\label{sec:conclusions_htoinv}

Constraining the Higgs boson to invisible state branching ratio is important in understanding whether it couples to dark matter at all, and if so how strongly. The measurement can also refute or restrict the viability of several dark matter models. From the analysis in Chpt.~\ref{chap:higgstoinv} utilising the full Run-2 dataset from \acrshort{cms}, a combined upper limit on \BRHinvFull of 28\,\% was observed at 95\,\% confidence level, and 20\,\% expected in the background-only hypothesis. While still far above the \acrlong{sm} prediction of 0.1\,\%, significant improvements have been made with respect to those obtained from Run-1 or 2016-only data.

The results with the full Run-2 dataset for both the \ttH and \VH channels set new, world-leading benchmarks for measuring the branching ratio of invisibly decaying Higgs bosons from these production modes. For \ttH, modest improvements are seen compared to the equivalent measurement from \acrshort{atlas}~\cite{ATLAS:2020kdi}, while large gains are achieved in comparison to the previous result by \acrshort{cms}~\cite{CMS-PAS-HIG-18-008}. An observed upper limit of 56\,\% and 50\,\% expected was achieved. Despite it being a more novel and challenging signature than some of the other production modes, this analysis has demonstrated that the sensitivity of the \ttH channel in hadronic final states is comparable to that of the cleaner leptonic channels and the other production modes of the Higgs boson.

For the \VH mechanism, there is no other public result (at the time of writing) encompassing both resolved and boosted topologies with a comparable dataset. Leveraging the full Run-2 dataset from \acrshort{cms} culminates in notably stronger sensitivity over previous results that analysed 2016 or Run-1 data: 22\,\% and 32\,\% upper limits were expected and observed, respectively. In addition to a superior limit in the \VH channel overall, the Run-2 limit from the 1V category---equivalent to a mono-\PVec phase space---sets a new precedent for sensitivity to the \higgstoinv decay.

Further improvements from a full combination over all Higgs production modes are inevitable. Given \acrshort{vbf} is the most sensitive channel---and leptonic final states such as Ref.~\citenum{CMS-PAS-HIG-18-008} have been discussed little---a grand combination with the other full Run-2 analyses will provide a much stronger limit. If there is a significant enhancement to the \acrshort{sm} value of $\BRof{\higgstoinv}$ due to a coupling to dark matter, it may be observable at the \acrshort{hllhc}. The volume of available data is expected to increase by more than ten times. Coupled with improvements to the \acrshort{cms} detector for improved particle identification, noise mitigation, higher bandwidth links to read out more detector information, and more precise measurements of particle properties, a promising result is within the realm of possibility. New and improved machine learning algorithms, for example to tag resolved decays of the top quark, and more generally to discriminate signal from background, will also contribute. In summary, advancements in several areas can each incrementally boost the sensitivity to this process.

Failing any enhancement from a dark matter coupling, observing the \acrshort{sm} rate of $\HepProcess{\PH \to \text{4}\Pnu}$ may require a precision-focused experiment. \acrshort{fcc}--$\Pe\mkern-1mu\Pe$ and the \acrshort{ilc} are strong candidates. Such ``Higgs factories'' would produce large quantities of Higgs bosons in $\APelectron \Pelectron$ collisions predominantly through the \VH and \acrshort{vbf} mechanisms.


%=========================================================


\section{Semi-visible jets}
\label{sec:conclusions_svj}

The search for \glspl{svj} pushes into unexplored territory. This novel array of theoretical models---with final states overlooked by most dark matter searches---provides a rich and diverse playground with many avenues to consider. Only a small fraction of signal models for the $s$- and \tchannel production mechanisms were analysed in Chpt.~\ref{chap:svj}. However, their characteristics suggest unique signatures compared to the expected \acrlong{sm} background and can be exploited with variables as simple as the transverse mass of the dijet system.

The analysis focused on the \schannel mode, performed by collaborators in \acrshort{cms}, is nearing completion at the time of writing. A dedicated search for the \tchannel mechanism has recently begun, as well as for \glspl{svj} originating from the decays of boosted \PZprime bosons. Any of these could yield important results with the full Run-2 dataset. With the aid of Run-3 and \acrshort{hllhc} data, and the other experimental improvements noted above, the boundaries will be further pushed. More advanced and versatile tagging algorithms could categorise a wide range of signal-like topologies while rejecting background to a larger degree. Higher energy future colliders, such as \acrshort{fcc}--hh would widen the search window. Mediator masses up to several tens of \TeVns could be probed, with a commensurate increase to the dark quark mass range. Additional topologies arising from variations of the $s$- and \tchannel modes, such as searches for displaced vertices/long-lived particles, open an even greater assortment of avenues to explore.
