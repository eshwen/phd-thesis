\section{Event selection}
\label{sec:htoinv_event_selection}

The event selection aims to strike a balance between rejecting as many background events while maintaining as great a signal sensitivity as possible. The preselection, in Chpt.~\ref{subsec:htoinv_preselection}, is applied to data and simulation in all regions and categories to do just that. Filters to reject potentially-mismeasured events and those that lead to incorrect \ptmiss calculations are documented in Chpt.~\ref{subsec:htoinv_other_filters}. A strategy to combat the HEM issue faced in 2018 (detailed in Chpt.~\ref{subsec:htoinv_data}) is given in Chpt.~\ref{subsec:htoinv_hem_mitigation}.

% This section is up-to-date as of 11th May


%=========================================================


\subsection{Preselection}
\label{subsec:htoinv_preselection}

The preselection is characterised by applying the following cuts:\footnote{Do I have to explain the reasoning behind each cut in the preselection?}
\medskip
\begin{easylist}[itemize]
    \cutflowlistprops
    & $\ptjone > \text{80} \GeV$
    & $\HT > \text{200} \GeV$
    & $\mht > \text{200} \GeV$
    & $\ptmiss > \text{200} \GeV$
    & $\mht/\ptmiss < \text{1.2}$ (inverted for some of the \acrshort{qcd}-enriched \glspl{SB})
    & Filters defined in Chpt.~\ref{subsec:htoinv_other_filters}
\end{easylist}

\medskip

\noindent{}To ensure orthogonality with the phase space occupied by their subset of the analysis, we also require $\abs{\etajone} < \text{2.4}$, $\abs{\etajtwo} < \text{2.4}$ if $\njet > \text{1}$, and a veto of the following \acrshort{vbf} kinematic selection:
\medskip
\begin{easylist}[itemize]
    \cutflowlistprops
    & $\ptjone > \text{80}\GeV$
    & $\ptjtwo > \text{40}\GeV$
    & $\abs{\etajone} < \text{5.0}$
    & $\abs{\etajtwo} < \text{5.0}$
    & $\etajone \cdot \etajtwo < \text{0}$
    & $\ptmiss \geq \text{250}\GeV$
    & $\abs{\Delta \eta_{\jone \jtwo}} > \text{1}$
    & $\mjj > \text{200}\GeV$
    & $\Delta \phi(\jone, \ \jtwo) < \text{1.5}$
    & $\mindphiAB{\mathrm{j}}{\ptmiss} > \text{0.5}$
\end{easylist}


%=========================================================


\subsection{Additional filters}
\label{subsec:htoinv_other_filters}

Further selections are applied to all years, regions and categories to filter poorly measured or mis-reconstructed events in both data and MC. A ``muon \gls{jet} filter'' rejects events with mis-reconstructed muons by requiring all \glspl{jet} with $\pt > \text{200}\GeV$ to have a muon energy fraction $f_{\mathrm{E}}^{\Pmu} < \text{0.5}$, and $\Delta\phi(\mathrm{j}, \ \ptmiss) < \pi - \text{0.4}$.

Charged ($f_{\mathrm{E}}^{\pm}$) and neutral hadron energy fraction ($f_{\mathrm{E}}^{0}$) requirements are applied to all \glspl{jet} via fulfillment of the ``tight'' \gls{jet} ID criteria (see Chpt.~\ref{subsec:objects_jets}). Furthermore, stricter selections are placed on the leading two \glspl{jet} as follows:

\medskip

\begin{easylist}[itemize]
    \cutflowlistprops
    & $f_{\mathrm{E}}^{\pm}(\jone) > \text{0.1}$
    & $f_{\mathrm{E}}^{0}(\jone) < \text{0.8}$
    & $f_{\mathrm{E}}^{\pm}(\jtwo) > \text{0.1}$
    & $f_{\mathrm{E}}^{0}(\jtwo) < \text{0.8}$
\end{easylist}

\medskip

\noindent{}Events with anomalously energetic \glspl{jet} have been observed for data in the \acrshort{hf} between $\abs{\eta} \in [\text{3.0}, \text{3.1}]$ that can lead to excess events, particularly in the \acrshort{qcd}-enriched \glspl{SB}. As such, a stipulation is placed on the ratio of the \HT formed from \glspl{jet} within $\abs{\eta} < \text{5}$ ($\HT^5$) and $\abs{\eta} < \text{2.4}$ ($\HT^{2.4}$), in OR with the difference in $\phi$ between the leading \gls{jet} and \mht:
\medskip
\begin{easylist}[itemize]
    \cutflowlistprops
    & $\dfrac{\HT^5}{\HT^{2.4}} < \text{1.2}$ \ or \ $\Delta\phi(\jone, \mht) \geq \left( \text{5.3} \dfrac{\HT^5}{\HT^{2.4}} - \text{4.78} \right)$  % dfrac for full-size fraction in list
\end{easylist}

\medskip

\noindent{}where, if the former requirement is not met, the latter is constructed such that $\Delta\phi \geq \pi/\text{2}$. The \HT used in the analysis (i.e, in the preselection) only places $\abs{\eta}$ circumscriptions on the two hardest \glspl{jet} from to the VBF orthogonality constraints in Chpt.~\ref{subsec:htoinv_preselection}.

To avoid any potential horns (spikes in data and/or \acrshort{mc}) for \glspl{jet} at large pseudorapidity, we require the agreement between the charged hadron-subtracted \MET and calorimeter-calculated \MET to be within 20\,\%:\footnote{Again, I try to use the symbol \ptmiss everywhere. But every instance of the CHS and calo MET has used \MET symbol, as far as I can tell. Should I swap to \ptmiss?}
\medskip
\begin{easylist}[itemize]
    \cutflowlistprops
    & $\text{1} - \dfrac{\MET(\text{CHS})}{\MET(\text{calo})} < \text{0.8}$
\end{easylist}

\medskip

\noindent{}The filters described below were recommended to remove events with miscalculated \ptmiss:
\medskip
\begin{easylist}[itemize]
    \easylistprops
    & Primary vertex filter to remove events failing vertex quality criteria
    & Beam halo filter
    & \acrshort{hcal} barrel and end cap noise filters
    & Filter for dead cells in the \acrshort{ecal} when constructing trigger primitives
    & Filter for low-quality \acrlong{pf} muons
\end{easylist}

\medskip

\noindent{}There are supplementary filters applied only to data. These are to generally mitigate \acrshort{ecal} end cap supercrystal noise, as well as crystals where losses of transparency would otherwise require large laser corrections.


%=========================================================


\subsection{Mitigating the HEM issue}
\label{subsec:htoinv_hem_mitigation}

In 2018, requirements were placed on electrons and the azimuthal angle of the \ptvecmiss to mitigate the HEM issue (see Chpt.~\ref{subsec:htoinv_data}). As such, the region-dependent selections for 2018 were updated to remove events that met the following:
\medskip
\begin{easylist}[itemize]
    \cutflowlistprops
    & $-\text{1.8} < \phi(\ptvecmiss) < -\text{0.6}$ in the signal region and \glspl{SB}
    & Any veto electron \vetoEle with $\pt > \text{10}\GeV$, $-\text{3.0} < \eta < -\text{1.4}$, and $-\text{1.57} < \phi < -\text{0.87}$ in the \singleEleCr \gls{CR}
    & Any loose photon \loosePhoton with $\pt > \text{15}\GeV$, $-\text{3.0} < \eta < -\text{1.4}$, and $-\text{1.57} < \phi < -\text{0.87}$ in the \singlePhotonCr \gls{CR}
\end{easylist}

\medskip

%\noindent{}Note that while the signal region and \acrshort{qcd} \glspl{SB} have hadronic final states, the implementation of veto weights for leptons in Chpt.~\ref{subsec:veto_sel_weights} does not automatically reject events containing leptons. To avoid ambiguity and the inclusion of potentially mis-reconstructed events, an explicit veto is placed for the electrons satisfying the above criteria.
% Put the above paragraph back in (and reword the paragraph before the list) if we end up using veto weights, and apply the electron and photon requirements to all regions
\noindent{}The effect of these cuts can be seen in Figs.~\ref{fig:htoinv_hem_issue_met_phi} and \ref{fig:htoinv_hem_issue_lepton_eta}.

\begin{figure}[htbp]
    \centering
    \begin{subfigure}[b]{0.34\textwidth}
        \includegraphics[width=\textwidth]{figures/hem_issue/sideband_4/met_phi_ttH_before_annotated.pdf}
        \caption{\ttH category}
    \end{subfigure}
    \hspace{0.05\textwidth}
    \begin{subfigure}[b]{0.34\textwidth}
        \includegraphics[width=\textwidth]{figures/hem_issue/sideband_4/met_phi_VH_before_annotated.pdf}
        \caption{\VH category}
    \end{subfigure}
    \caption[The azimuthal angle of the \ptvecmiss in the \ttH and \VH categories before applying the selections designed to mitigate the HEM issue in 2018]{The azimuthal angle of the \ptvecmiss in the \ttH and \VH categories before applying the selections designed to mitigate the HEM issue in 2018. The loose \omegaTilde \gls{SB} is used to demonstrate the effect since it kinematically resembles the signal region, and the data--simulation discrepancy can be removed while still blind in said region. A red box encloses the sector that is removed by the selection applied in the signal region and \glspl{SB}.}
    \label{fig:htoinv_hem_issue_met_phi}
\end{figure}

\begin{figure}[htbp]
    \centering
    \begin{subfigure}[b]{0.34\textwidth}
        \includegraphics[width=\textwidth]{figures/hem_issue/region_3/leadLepton_eta_VH_before.pdf}
        \caption{$\eta_{\Pe}$ before cut}
    \end{subfigure}
    \hspace{0.05\textwidth}
    \begin{subfigure}[b]{0.34\textwidth}
        \includegraphics[width=\textwidth]{figures/hem_issue/region_3/leadLepton_eta_VH_after.pdf}
        \caption{$\eta_{\Pe}$ after cut}
    \end{subfigure}
    \caption[The psuedorapidity of the electron in the \VH category of the \singleEleCr \gls{CR} before applying the selection designed to mitigate the HEM issue in 2018]{The psuedorapidity of the electron in the \VH category of the \singleEleCr \gls{CR} before applying the selection designed to mitigate the HEM issue in 2018.}
    \label{fig:htoinv_hem_issue_lepton_eta}
\end{figure}

% Figures from 29th May, 2020
