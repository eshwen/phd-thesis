\section{Statistical model for fit to data}
\label{sec:htoinv_satistical_treatment}

A likelihood model is used in the fit to data, simultaneously over the signal and \glspl{CR} to obtain the \acrlong{sm} expectation values as well as testing for signals of the \higgstoinv decay. Events are categorised in two dimensions: the categories that target specific Higgs boson production modes and final state topologies as per Tab.~\ref{tab:htoinv_categories}, and in bins of \ptmiss as per Tab.~\ref{tab:htoinv_binning_scheme}. The observed event counts from data in each category and \ptmiss bin are modelled as Poisson-distributed variables around the \acrshort{sm} expectation with a potential contribution from signal (assumed to be zero in the null hypothesis). Expected event counts in the signal region are obtained from simulation, aided by corrections from the \glspl{CR} and \glspl{SB} for electroweak and \acrshort{qcd} multijet processes, respectively.

In the signal region, the background samples in Sec.~\ref{subsec:htoinv_background} are grouped into three processes for the purposes of background estimation in the fit:
\medskip
\begin{easylist}[itemize]
    \easylistprops
    & Lost lepton: Comprised of \ttbarpjets, \wtolnupjets, single top, $\ttbar\Pphoton \plusjets$, $\ttbar\PW \plusjets$, $\ttbar\PH \plusjets$, electroweak $\PW + \text{2 jets}$, and \gammapjets 
    & $\ztonunu$: Comprised of \ztonunupjets, multiboson, electroweak $\PZ + \text{2 jets}$, $\ttbar\PZ \plusjets$, and Drell-Yan $(\HepProcess{\PZ \to \Plepton\Plepton}) \plusjets$
    & \acrshort{qcd}: \acrshort{qcd} multijet
\end{easylist}
\medskip
\noindent{}Systematic uncertainties associated with simulation of both signal and background processes, discussed in Sec.~\ref{sec:htoinv_mc_corrections}, are incorporated as nuisance parameters within the model. The likelihood function $\likelihood_{\higgstoinv}$ can be summarised as
\begin{equation}
    \likelihood_{\higgstoinv} = \likelihood_{\text{SR}} \cdot \likelihood_{\text{\singleMuCr CR}} \cdot \likelihood_{\text{\doubleMuCr CR}} \cdot \likelihood_{\text{\singleEleCr CR}} \cdot \likelihood_{\text{\doubleMuCr CR}} \cdot \likelihood_{\text{\singlePhotonCr CR}},
    \label{eq:likelihood_overall}
\end{equation}
where the aim of the fit is to minimise $-\ln \likelihood_{\higgstoinv}$. The likelihood in a given region of the analysis (where the subscript below $\likelihood$ indicates said region) may be written as multiple Poisson likelihoods, denoting $\mathcal{P}(n | \lambda) \equiv \frac{ e^{-\lambda} \lambda^n }{n!}$. In the signal region,
\begin{equation}
    \begin{aligned}
\likelihood_{\text{SR}}(r, a_{\lostlepton}, a_{\ztonunu}, \rho) &= \prod_{i} \prod_{j(i)} \mathcal{P}(N_{\mathrm{obs.}}^{i, j} | N_{\mathrm{pred.}}^{i, j}), \ \text{where}\\
N_{\mathrm{pred.}}^{i, j} &= r \cdot s^{i, j} \cdot \rho_s^{i, j}\\
&+ b_{\lostlepton}^{i, j} \cdot a_{\lostlepton}^{i, j} \cdot \rho_{\lostlepton}^{i, j}\\
&+ b_{\ztonunu}^{i, j} \cdot a_{\ztonunu}^{i, j} \cdot \rho_{\ztonunu}^{i, j}\\
&+ c_{\mathrm{QCD}}^{i, j} \cdot \omega_{\mathrm{QCD}}^{i, j},
    \end{aligned}
    \label{eq:likelihood_SR}
\end{equation}
where the indices $i$ and $j$ refer to each category and \ptmiss bin, respectively, $r$ is the unconstrained signal strength parameter, i.e., \BRHinvFull, $s$ is the signal expectation determined from simulation, $\rho$ encodes the systematic uncertainties associated with simulation as Gaussian constraints, $b$ is the number of events from simulation, $a$ is an initially unconstrained rate parameter (extracted in the fit) connecting the signal and corresponding \glspl{CR}, $c$ is the number of \acrshort{qcd} multijet events, and $\omega$ contains the uncertainties on those events with Gaussian constraints. Similarly for the \glspl{CR}, the likelihood functions are
\begin{equation}
    \begin{aligned}
\likelihood_{\text{\singleMuCr CR}} &= \prod_{i} \prod_{j(i)} \mathcal{P}( N_{\mathrm{obs.}, \, \Pmu }^{i, j} | r \cdot s^{i, j}_{\Pmu} \cdot \rho_{s,\, \Pmu}^{i, j} + b_{\Pmu}^{i, j} \cdot a_{\lostlepton}^{i, j} \cdot \rho_{\Pmu}^{i, j} ),\\
\likelihood_{\text{\doubleMuCr CR}} &= \prod_{i} \prod_{j(i)} \mathcal{P}( N_{\mathrm{obs.}, \, \Pmu\Pmu }^{i, j} | r \cdot s^{i, j}_{\Pmu\Pmu} \cdot \rho_{s,\, \Pmu\Pmu}^{i, j} + b_{\Pmu\Pmu}^{i, j} \cdot a_{\ztonunu}^{i, j} \cdot \rho_{\Pmu\Pmu}^{i, j} ),\\
\likelihood_{\text{\singleEleCr CR}} &= \prod_{i} \prod_{j(i)} \mathcal{P}( N_{\mathrm{obs.}, \, \Pe }^{i, j} | r \cdot s^{i, j}_{\Pe} \cdot \rho_{s,\, \Pe}^{i, j} + b_{\Pe}^{i, j} \cdot a_{\lostlepton}^{i, j} \cdot \rho_{\Pe}^{i, j} ),\\
\likelihood_{\text{\doubleEleCr CR}} &= \prod_{i} \prod_{j(i)} \mathcal{P}( N_{\mathrm{obs.}, \, \Pe\Pe }^{i, j} | r \cdot s^{i, j}_{\Pe\Pe} \cdot \rho_{s,\, \Pe\Pe}^{i, j} + b_{\Pe\Pe}^{i, j} \cdot a_{\ztonunu}^{i, j} \cdot \rho_{\Pe\Pe}^{i, j} ),\\
\likelihood_{\text{\singlePhotonCr CR}} &= \prod_{i} \prod_{j(i)} \mathcal{P}( N_{\mathrm{obs.}, \, \Pphoton }^{i, j} | r \cdot s^{i, j}_{\Pphoton} \cdot \rho_{s,\, \Pphoton}^{i, j} + b_{\Pphoton}^{i, j} \cdot a_{\ztonunu}^{i, j} \cdot \rho_{\Pphoton}^{i, j} + c_{\mathrm{QCD}}^{i, j} \cdot \omega_{\mathrm{QCD}}^{i, j} ),
    \end{aligned}
    \label{eq:likelihood_CRs}
\end{equation}
where the products over the indices $i$ and $j$ are the same as in Eq.~\ref{eq:likelihood_SR}. Signal contamination $s$ is accounted for in all \glspl{CR}. The rate parameters $a$ are shared across the signal region and complementary \glspl{CR} for the same categories and \ptmiss bins, i.e., $a_{\lostlepton}$ in the single lepton regions, and $a_{\ztonunu}$ in the dilepton and single photon regions. All non-multijet samples are grouped into a single process, represented by $b$.\footnote{In pre-fit or post-fit figures, the non-multijet background is labelled as ``electroweak.''} \acrshort{qcd} multijet \acrshort{mc} is not included in any of the lepton \glspl{CR}, but is estimated in the \singlePhotonCr \gls{CR} from a photon purity measurement discussed in Sec.~\ref{subsubsec:htoinv_photon_purity}. As in the signal region, the number of \acrshort{qcd} events is given by $c$, with the associated systematic uncertainty (applied as a Gaussian constraint) represented by $\omega$.

Three types of fit to data are considered in the analysis: one involving only the \glspl{CR}---the \emph{\gls{CR}--only} fit---where the likelihood function only comprises the terms in Eq.~\ref{eq:likelihood_CRs}; a combined fit to the \glspl{CR}, and the background simulation and data in the signal region, corresponding to the \emph{background--only} hypothesis; and the fit that also includes signal simulation in the signal region, corresponding to the \emph{signal-plus-background} hypothesis. The \gls{CR}--only fit is used to judge the level to which the \acrshort{sm} background prediction from those regions describes the observed data in the signal region, without the knowledge to constrain it directly. A metric for optimising the sensitivity of an analysis is the value of the expected upper limit on the signal strength parameter resulting from this fit. In the absence of a statistically significant excess of events in the signal region, the background--only fit constrains the upper limit on $\BRof{\higgstoinv}$ expected in the \acrshort{sm}, and the result of the signal-plus-background fit is the observed limit.

The \CLs method for setting an upper limit on $r$ is used in the case no new physics is observed. It is often applied to set exclusion limits on non-negative parameters. Instead of simply comparing the $p$-value from the signal-plus-background (alternative) hypothesis $p_{s+b}$---obtained from the fit---to the threshold $\alpha$, both $p_{s+b}$ and the $p$-value from the background--only (null) hypothesis $p_b$ are involved. The alternative hypothesis is rejected, i.e., the signal model is excluded, where
\begin{equation}
    \frac{p_{s+b}}{1 - p_b} \leq \alpha.
    \label{eq:CLs_ratio}
\end{equation}
As such, the upper limit on the signal strength parameter is given by the value of $r$ in which the left-hand side of Eq.~\ref{eq:CLs_ratio} equals $\alpha$. An asymptotic formula~\cite{Cowan:2010js}, whereby the ensemble of simulated datasets is replaced by the single, representative one to reduce computational expense, is applicable in the large sample limit and is incorporated into the model.

Statistical uncertainties from simulation are accommodated as a single nuisance parameter per bin based on Ref.~\citenum{BARLOW1993219}. Above 10 weighted events, the uncertainty is profiled according to a Gaussian distribution centred on its value. Below that threshold, a Poisson distribution is instead invoked to provide stability in the fit for bins with small event counts. These methods are implemented, along with the likelihood function, in the \textsf{HiggsAnalysis-CombinedLimit} package.

% Current fit model is given at https://indico.cern.ch/event/934008/contributions/3924639/attachments/2065231/3465892/2020_06_15_CHIP_Meeting_nonVBF_fit.pdf

% Toys (verbatim from Henning): toys are repeat experiments where you draw new values for your observed signal and background based on the uncertainties associated with both. The parameters values are randomly drawn according to some pdf. I.e. if we talk about large event counts this would be gaussian with width equal to uncertainty,  small event counts Poissonian, for syst uncert possibly log-normal.
