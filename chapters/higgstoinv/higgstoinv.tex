\chapter{Search for invisibly decaying Higgs bosons}
\label{chap:higgstoinv}

\epigraph{Invisibility---there are things we can't see now, that are there, that are embedded, that it really takes time in order to be able to see. There are many ghosts that are lurking around and lingering through us that takes the technology of another generation or so in order to uncover and show what those stains and strains and perceived flaws really we're building towards.}{--- Lynn Hershman Leeson}

\initial{P}articles that escape the detector unseen in any experiment make them, by design, notoriously difficult to search for. The Higgs boson is particularly challenging with its small production rate at the \acrshort{lhc} and a commensurate prediction of the invisible state branching ratio. As described in Sec.~\ref{sec:theory_higgs_to_inv}, the leading experimental upper limits are still far higher than the \acrlong{sm}'s value. For the best chance of observing this decay, the inclusion of more than one of the Higgs boson's production modes is a necessity.


%=========================================================


\import{./}{analysis_overview.tex}


%=========================================================


\import{./}{software.tex}


%=========================================================


\import{./}{data_simulation.tex}


%=========================================================


\import{./}{event_selection.tex}


%=========================================================


\import{./}{categorisation.tex}


%=========================================================


\import{./}{fit.tex}


%=========================================================


\import{./}{background_estimation.tex}


%=========================================================


\import{./}{weights_and_systs.tex}


%=========================================================


\import{./}{results.tex}


%=========================================================


\import{./}{interpretations.tex}
