\section{Results}
\label{sec:htoinv_results}

The fits of signal and background to the observed data are performed as outlined in the previous section. In the following subsections, the results of the fits for each category and data taking period are shown in several forms: pre-fit vs. post-fit distributions in each subcategory and \ptmiss bin; the observed upper limit on the signal strength parameter $\BRHinvFull$ at 95\,\% confidence level from the signal plus background hypothesis, accompanied by the median expected limit with 68\,\% and 95\,\% confidence level intervals from the background-only hypothesis; and profile likelihood ratios as a function of the signal strength.


%=========================================================


\subsection{Results of the \texorpdfstring{\ttH}{ttH} analysis}
\label{subsec:htoinv_analysis_ttH}

The \singlePhotonCr \gls{CR} is not used in constraining the \ztonunu prediction for any of the subcategories. To predict the \acrshort{qcd} event counts in the signal region, Eq.~\ref{eq:qcd_prediction} incorporates the \acrshort{mc} counts in said region and from the loose double sideband from Tab.~\ref{tab:sideband_defs_ttH}---the most enriched in multijet \acrshort{mc}. The category and \ptmiss fractions, $N_{\mathrm{SB}}^{\mathrm{QCD}}$, and $\transfac_{\mathrm{QCD}}$ were derived from the sideband.

Distributions of the post-fit yields in the signal region for 2016, 2017, and 2018 are displayed in Fig.~\ref{fig:htoinv_mountain_range_ttH_SR}. The total pre-fit background is also overlaid. Corresponding figures for the \glspl{CR} are given in App.~\ref{sec:pre_post_fit_plots_ttH_CRs}.\footnote{Add at least a paragraph discussing the pre-fit vs post-fit yields, i.e., how well the fit is able to model the background/allow it to fill any pre-fit data excesses, once I have the final results.}

\begin{figure}[htbp]
    \centering
    \begin{subfigure}[b]{0.9\textwidth}
        \includegraphics[width=\textwidth]{figures/mountain_ranges/2016/ttH/SR_tree_fit_s-abs_values_ttH_cats.pdf}
        \caption{\ttH --- 2016}
    \end{subfigure}

    \begin{subfigure}[b]{0.9\textwidth}
        \includegraphics[width=\textwidth]{figures/mountain_ranges/2017/ttH/SR_tree_fit_s-abs_values_ttH_cats.pdf}
        \caption{\ttH --- 2017}
    \end{subfigure}

    \begin{subfigure}[b]{0.9\textwidth}
        \includegraphics[width=\textwidth]{figures/mountain_ranges/2018/ttH/SR_tree_fit_s-abs_values_ttH_cats.pdf}
        \caption{\ttH --- 2018}
    \end{subfigure}
    \caption[Pre-fit and post-fit yields in the signal region for each \ttH subcategory and \ptmiss bin in each year of Run-2]{Pre-fit and post-fit yields in the signal region for each \ttH subcategory and \ptmiss bin in each year of Run-2.}
    \label{fig:htoinv_mountain_range_ttH_SR}
\end{figure}

It is evident from the distributions that the fit succeeds in most cases to match the signal and background to the data within uncertainties. Relatively few bins contain a post-fit data/prediction ratio largely different from unity. Large deviations are seen only in the \ttH boosted subcategories which are statistically limited. In 2016, a general over-prediction of background is seen, corrected for most part by the fit. The pre-fit ratio is overall better in 2016 than in the other years. The effect of pre-firing was less severe than in 2017. Both of the later years suffered from additional problems such as noise in the \acrshort{ecal} end caps in 2017---affecting \ttH given its high jet multiplicity---and the HEM issue in 2018 that largely affects \glspl{jet}. Techniques were employed to mitigate the problems, but of course may have failed to completely eradicate them.

These post-fit distributions translate directly into the upper limit on \BRHinvFull. Fig.~\ref{fig:htoinv_limit_ttH} showcases the limit and profile likelihood scan for the \ttH category in each data taking year individually and the combination over the full Run-2 dataset. Limits broken down by subcategory in each year are presented in Fig.~\ref{fig:htoinv_limit_ttH_per_year}.\footnote{Add at least a paragraph discussing the limits, and their consistency, once I have the final results. Indicate problems and stuff. Relate limits to the mountain ranges.}

\begin{figure}[htbp]
    \centering
    \begin{subfigure}[t]{0.45\textwidth}
        \includegraphics[width=\textwidth]{figures/limits/ttH/limit_Run2_ttH.pdf}
        \caption{Limit --- \ttH}
    \end{subfigure}
    \hspace{0.05\textwidth}
    \begin{subfigure}[t]{0.45\textwidth}
        \includegraphics[width=\textwidth]{figures/likelihood_scan/profile_likelihood_scan_Run2_ttH.pdf}
        \caption{Profile likelihood --- \ttH}
    \end{subfigure}
    \caption[Observed and expected 95\,\% CL upper limit on the Higgs boson to invisible state branching fraction $\BRof{\higgstoinv}$ (left) and the corresponding profile likelihood ratio as a function of it (right) in the \ttH category]{Observed and expected 95\,\% CL upper limit on the Higgs boson to invisible state branching fraction $\BRof{\higgstoinv}$ (left) and the corresponding profile likelihood ratio as a function of it (right) in the \ttH category. The result from each data taking period is presented along with their combination.}
    \label{fig:htoinv_limit_ttH}
\end{figure}

\textbf{Talk about the limit and likelihood scan. The overall combined limit is XX\%, much better than ATLAS' hadronic ttH. Compare observed vs expected. Compare our 2016 limit with the published one. Note that differences could be due to systematic uncertainties (e.g., we include QCD scale systs that they don't, and possibly others). Differences in MC could also have an effect. Talk about the likelihood scans briefly. The median expected limit can be seen from the intersection of the dashed curve with the horizontal line $-\text{2}\Delta \ln(\likelihood) = \text{3.84}$ (i.e., at the 95\% confidence level interval). The intersection of the solid curve with the same horizontal line gives the observed limit at 95\,\% CL.}

\clearpage


%=========================================================


\subsection{Results of the \texorpdfstring{\VH}{VH} analysis}
\label{subsec:htoinv_analysis_VH}

Contrary to \ttH, all of the \glspl{CR} are utilised in the non-multijet background predictions. A more accurate \ztonunu prediction is possible since it can be constrained by three \glspl{CR}. The single sideband inverted in \mindphi, \omegaTilde, and dijet mass was applied to predict the \acrshort{qcd} contribution to the signal region in all \VH subcategories.

Distributions of the pre-fit and post-fit yields for 2016, 2017, and 2018 are displayed in Fig.~\ref{fig:htoinv_mountain_range_VH_SR}. Corresponding figures for the \glspl{CR} are given in App.~\ref{sec:pre_post_fit_plots_VH_CRs}.\footnote{Add at least a paragraph discussing the pre-fit vs post-fit yields, i.e., how well the fit is able to model the background/allow it to fill any pre-fit data excesses, once I have the final results.}

\begin{figure}[htbp]
    \centering
    \begin{subfigure}[b]{0.9\textwidth}
        \includegraphics[width=\textwidth]{figures/mountain_ranges/2016/VH/SR_tree_fit_s-abs_values_VH_cats.pdf}
        \caption{\VH --- 2016}
    \end{subfigure}

    \begin{subfigure}[b]{0.9\textwidth}
        \includegraphics[width=\textwidth]{figures/mountain_ranges/2017/VH/SR_tree_fit_s-abs_values_VH_cats.pdf}
        \caption{\VH --- 2017}
    \end{subfigure}

    \begin{subfigure}[b]{0.9\textwidth}
        \includegraphics[width=\textwidth]{figures/mountain_ranges/2018/VH/SR_tree_fit_s-abs_values_VH_cats.pdf}
        \caption{\VH --- 2018}
    \end{subfigure}
    \caption[Pre-fit and post-fit yields in the signal region for each \VH subcategory and \ptmiss bin in each year of Run-2]{Pre-fit and post-fit yields in the signal region for each \VH subcategory and \ptmiss bin in each year of Run-2.}
    \label{fig:htoinv_mountain_range_VH_SR}
\end{figure}

Fig.~\ref{fig:htoinv_limit_VH} showcases the expected and observed limits for the \VH category in each year and the Run-2 combination. Limits for each subcategory can be found in Fig.~\ref{fig:htoinv_limit_VH_per_year}.\footnote{Add at least a paragraph discussing the limits, and their consistency, once I have the final results. Indicate problems and stuff. Relate limits to the mountain ranges.}

\begin{figure}[htbp]
    \centering
    \begin{subfigure}[t]{0.45\textwidth}
        \includegraphics[width=\textwidth]{figures/limits/VH/limit_Run2_VH.pdf}
        \caption{Limit --- \VH}
    \end{subfigure}
    \hspace{0.05\textwidth}
    \begin{subfigure}[t]{0.45\textwidth}
        \includegraphics[width=\textwidth]{figures/likelihood_scan/profile_likelihood_scan_Run2_VH.pdf}
        \caption{Profile likelihood --- \VH}
    \end{subfigure}
    \caption[Observed and expected 95\,\% CL upper limit on the Higgs boson to invisible state branching fraction $\BRof{\higgstoinv}$ (left) and the corresponding profile likelihood ratio as a function of it (right) in the \VH category]{Observed and expected 95\,\% CL upper limit on the Higgs boson to invisible state branching fraction $\BRof{\higgstoinv}$ (left) and the corresponding profile likelihood ratio as a function of it (right) in the \VH category. The result from each data taking period is presented along with their combination.}
    \label{fig:htoinv_limit_VH}
\end{figure}

\clearpage


%=========================================================


\section{Combined results}
\label{sec:htoinv_combined_results}

% Show the results combined over all production modes (one plot for each year), then the full combination for Run-2, ideally with VBF results as well

Upper limits for $\BRof{\higgstoinv}$ by combining all categories for the full Run-2 dataset can be shown as broken down by data taking year in Fig.~\ref{fig:htoinv_limit_likelihood_Run2_per_year} and by category in Fig.~\ref{fig:htoinv_limit_likelihood_Run2_per_cat}. Profile likelihood ratios as a function of $\BRof{\higgstoinv}$ are also presented opposite the limits. Corresponding limits and likelihood ratios for each year are displayed in Figs.~\ref{fig:htoinv_limit_likelihood_2016}, \ref{fig:htoinv_limit_likelihood_2017}, \ref{fig:htoinv_limit_likelihood_2018}, respectively, of App.~\ref{sec:limits_likelihoods_year_supplementary}.\footnote{Add some reasonable discussion of the limits and likelihoods, consistency across categories/years, once I have the final results.}\footnote{If we see worse results in 2016 compared to previous analyses, one explainer could be the signal models. It was recently discovered that signal MC used for 2016 analyses had a harder pt spectrum than what has been used in this analysis, which would give improved limits there as the distributions are shifted to higher MET and giving better signal sensitivity.}

\begin{figure}[htbp]
    \centering
    \begin{subfigure}[t]{0.45\textwidth}
        \includegraphics[width=\textwidth]{figures/limits/full_Run2/limit_Run2_comb_per_year.pdf}
        \caption{Limit --- Run-2}
    \end{subfigure}
    \hspace{0.05\textwidth}
    \begin{subfigure}[t]{0.45\textwidth}
        \includegraphics[width=\textwidth]{figures/likelihood_scan/profile_likelihood_scan_Run2_per_year.pdf}
        \caption{Profile likelihood --- Run-2}
    \end{subfigure}
    \caption[Observed and expected 95\,\% CL upper limit on the Higgs boson to invisible state branching fraction $\BRof{\higgstoinv}$ and the corresponding profile likelihood ratio as a function of it, for both the individual data taking years, as well as the combination of them, for the full Run-2 dataset]{Observed and expected 95\,\% CL upper limit on the Higgs boson to invisible state branching fraction $\BRof{\higgstoinv}$ (left) and the corresponding profile likelihood ratio as a function of it (right), for both the individual data taking years, as well as the combination of them, for the full Run-2 dataset. The \acrlong{sm} Higgs boson with its associated mass and production cross section are assumed.}
    \label{fig:htoinv_limit_likelihood_Run2_per_year}
\end{figure}

\begin{figure}[htbp]
    \centering
    \begin{subfigure}[t]{0.45\textwidth}  % top align since axis labels are larger for likelihood
        \includegraphics[width=\textwidth]{figures/limits/full_Run2/limit_Run2_comb_per_cat.pdf}
        \caption{Limit --- Run-2}
    \end{subfigure}
    \hspace{0.05\textwidth}
    \begin{subfigure}[t]{0.45\textwidth}
        \includegraphics[width=\textwidth]{figures/likelihood_scan/profile_likelihood_scan_Run2_per_cat.pdf}
        \caption{Profile likelihood --- Run-2}
    \end{subfigure}
    \caption[Observed and expected 95\,\% CL upper limit on the Higgs boson to invisible state branching fraction $\BRof{\higgstoinv}$ and the corresponding profile likelihood ratio as a function of it, for both the individual categories, as well as the combination of them, for the full Run-2 dataset]{Observed and expected 95\,\% CL upper limit on the Higgs boson to invisible state branching fraction $\BRof{\higgstoinv}$ (left) and the corresponding profile likelihood ratio as a function of it (right), for both the individual categories, as well as the combination of them, for the full Run-2 dataset. The \acrlong{sm} Higgs boson with its associated mass and production cross section are assumed.}
    \label{fig:htoinv_limit_likelihood_Run2_per_cat}
\end{figure}

\begin{table}[htbp]
    \centering
    \begin{tabular}{ccccc}
        \hline\hline
        Dataset & \ttH & \VH & Combined\\\hline
        \multirow{2}{*}{2016} & 89\,\% (obs.) & 22\,\% (obs.) & 23\,\% (obs.) \\
        & 80\,\% (exp.) & 41\,\% (exp.) & 36\,\% (exp.) \\\hline
        \multirow{2}{*}{2017} & 116\,\% (obs.) & 43\,\% (obs.) & 42\,\% (obs.) \\
        & 90\,\% (exp.) & 38\,\% (exp.) & 36\,\% (exp.) \\\hline
        \multirow{2}{*}{2018} & 98\,\% (obs.) & 73\,\% (obs.) & 68\,\% (obs.) \\
        & 80\,\% (exp.) & 34\,\% (exp.) & 31\,\% (exp.) \\\hline
        \multirow{2}{*}{Run-2} & 57\,\% (obs.) & 32\,\% (obs.) & \textbf{28\,\% (obs.)} \\
        & 50\,\% (exp.) & 22\,\% (exp.) & \textbf{20\,\% (exp.)} \\\hline\hline
    \end{tabular}
    \caption[Observed and median expected upper limits on $\BRof{\higgstoinv}$ for each combination of category and dataset in the analysis]{Observed and median expected upper limits on $\BRof{\higgstoinv}$ for each combination of category and dataset in the analysis.}
    \label{tab:hinv_limits}
\end{table}
