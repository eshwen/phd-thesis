\section{Analysis of the \texorpdfstring{\ttH}{ttH} mode}
\label{sec:htoinv_analysis_ttH}

The fit of signal and background to the observed data is performed for each subcategory of \ttH, in addition to the category as a whole.\footnote{Might make more sense to have a section called ``Results'', and then these ``Analysis of the X mode'' and combined results as subsections.} Due to limited statistical precision, the dilepton \glspl{CR} in the boosted subcategories are combined into a single bin, i.e., inclusive of both subcategory and \ptmiss. The rate parameter for the prediction of the \ztonunu background in the signal region is therefore a single value shared across the boosted subcategories and their \ptmiss bins. The \singlePhotonCr \gls{CR} is not used in constraining the \ztonunu prediction for any of the subcategories.

To predict the \acrshort{qcd} event counts in the signal region, Eq.~\ref{eq:qcd_prediction} incorporates the \acrshort{mc} counts in said region and from one or more sidebands. The tight double sideband from Tab.~\ref{tab:sideband_defs}---the most enriched in multijet \acrshort{mc}---is employed to derive the category fraction \catFraction. Yields from the loose double sideband compute the remaining terms: the \ptmiss fraction \metFraction, $N_{\mathrm{SB}}^{\mathrm{QCD}}$, and $\transfac_{\mathrm{QCD}}$.

Distributions of the post-fit yields in the signal region for 2016, 2017, and 2018 are displayed in Fig.~\ref{fig:htoinv_mountain_range_ttH_SR}. The total pre-fit background is also overlaid. Corresponding figures for the \glspl{CR} are given in App.~\ref{sec:pre_post_fit_plots_ttH_CRs}.\footnote{Add at least a paragraph discussing the pre-fit vs post-fit yields, i.e., how well the fit is able to model the background/allow it to fill any pre-fit data excesses, once I have the final results.}

\begin{figure}[htbp]
    \centering
    \begin{subfigure}[b]{0.9\textwidth}
        \includegraphics[width=\textwidth]{figures/mountain_ranges/2016/ttH/SR_tree_fit_s-abs_values_ttH_cats.pdf}
        \caption{\ttH --- 2016}
    \end{subfigure}

    \begin{subfigure}[b]{0.9\textwidth}
        \includegraphics[width=\textwidth]{figures/mountain_ranges/2017/ttH/SR_tree_fit_s-abs_values_ttH_cats.pdf}
        \caption{\ttH --- 2017}
    \end{subfigure}

    \begin{subfigure}[b]{0.9\textwidth}
        \includegraphics[width=\textwidth]{figures/mountain_ranges/2018/ttH/SR_tree_fit_s-abs_values_ttH_cats.pdf}
        \caption{\ttH --- 2018}
    \end{subfigure}
    \caption[Pre-fit and post-fit yields in the signal region for each \ttH subcategory and \ptmiss bin in each year of Run-2]{Pre-fit and post-fit yields in the signal region for each \ttH subcategory and \ptmiss bin in each year of Run-2.}
    \label{fig:htoinv_mountain_range_ttH_SR}
\end{figure}

Results of the fit tot data are presented in terms of the upper limit on the signal strength parameter \BRHinvFull. The observed limit in the signal plus background hypothesis at 95\,\% confidence level is overlaid on the expectation from the background-only hypothesis. In the latter, the median expected limit is illustrated with accompanying boundaries for the 68\,\% and 95\,\% confidence level intervals. Fig.~\ref{fig:htoinv_limit_ttH} showcases the limits for each \ttH subcategory and the combination of them all, for each data taking year individually and over the full Run-2 dataset.\footnote{Add at least a paragraph discussing the limits, and their consistency, once I have the final results. Indicate problems and stuff. Relate limits to the mountain ranges.}

\begin{figure}[htbp]
    \centering
    \begin{subfigure}[b]{0.45\textwidth}
        \includegraphics[width=\textwidth]{figures/limits/ttH/limit_2016_ttH.pdf}
        \caption{\ttH --- 2016}
    \end{subfigure}
    \hfill
    \begin{subfigure}[b]{0.45\textwidth}
        \includegraphics[width=\textwidth]{figures/limits/ttH/limit_2017_ttH.pdf}
        \caption{\ttH --- 2017}
    \end{subfigure}

    \begin{subfigure}[b]{0.45\textwidth}
        \includegraphics[width=\textwidth]{figures/limits/ttH/limit_2018_ttH.pdf}
        \caption{\ttH --- 2018}
    \end{subfigure}
    \hfill
    \begin{subfigure}[b]{0.45\textwidth}
        \includegraphics[width=\textwidth]{figures/limits/ttH/limit_Run2_ttH.pdf}
        \caption{\ttH --- Run-2}
    \end{subfigure}
    \caption[Observed and expected 95\,\% CL upper limits on the Higgs boson to invisible state branching fraction in the \ttH category, for both the individual subcategories, and the combination of them, for each data-taking year in Run-2]{Observed and expected 95\,\% CL upper limits on the Higgs boson to invisible state branching fraction in the \ttH category, for both the individual subcategories, and the combination of them, for each data-taking year in Run-2.}
    \label{fig:htoinv_limit_ttH}
\end{figure}

\clearpage


%=========================================================


\section{Analysis of the \texorpdfstring{\VH}{VH} mode}
\label{sec:htoinv_analysis_VH}

Contrary to \ttH, all of the \glspl{CR} are utilised in the non-multijet background predictions with a fully granular correspondence to the signal region.\footnote{Add a sentence or two about how QCD prediction is done. Might be different from ttH.}

Distributions of the pre-fit and post-fit yields for 2016, 2017, and 2018 are displayed in Fig.~\ref{fig:htoinv_mountain_range_VH_SR}. Corresponding figures for the \glspl{CR} are given in App.~\ref{sec:pre_post_fit_plots_VH_CRs}.\footnote{Add at least a paragraph discussing the pre-fit vs post-fit yields, i.e., how well the fit is able to model the background/allow it to fill any pre-fit data excesses, once I have the final results.}

\begin{figure}[htbp]
    \centering
    \begin{subfigure}[b]{0.9\textwidth}
        \includegraphics[width=\textwidth]{figures/mountain_ranges/2016/VH/SR_tree_fit_s-abs_values_VH_cats.pdf}
        \caption{\VH --- 2016}
    \end{subfigure}

    \begin{subfigure}[b]{0.9\textwidth}
        \includegraphics[width=\textwidth]{figures/mountain_ranges/2017/VH/SR_tree_fit_s-abs_values_VH_cats.pdf}
        \caption{\VH --- 2017}
    \end{subfigure}

    \begin{subfigure}[b]{0.9\textwidth}
        \includegraphics[width=\textwidth]{figures/mountain_ranges/2018/VH/SR_tree_fit_s-abs_values_VH_cats.pdf}
        \caption{\VH --- 2018}
    \end{subfigure}
    \caption[Pre-fit and post-fit yields in the signal region for each \VH subcategory and \ptmiss bin in each year of Run-2]{Pre-fit and post-fit yields in the signal region for each \VH subcategory and \ptmiss bin in each year of Run-2.}
    \label{fig:htoinv_mountain_range_VH_SR}
\end{figure}

Fig.~\ref{fig:htoinv_limit_VH} showcases the expected and observed limits for the \VH subcategories and their combination for each data taking year as well the result for the full Run-2 dataset.\footnote{Add at least a paragraph discussing the limits, and their consistency, once I have the final results. Indicate problems and stuff. Relate limits to the mountain ranges.}

\begin{figure}[htbp]
    \centering
    \begin{subfigure}[b]{0.45\textwidth}
        \includegraphics[width=\textwidth]{figures/limits/VH/limit_2016_VH.pdf}
        \caption{\VH --- 2016}
    \end{subfigure}
    \hfill
    \begin{subfigure}[b]{0.45\textwidth}
        \includegraphics[width=\textwidth]{figures/limits/VH/limit_2017_VH.pdf}
        \caption{\VH --- 2017}
    \end{subfigure}

    \begin{subfigure}[b]{0.45\textwidth}
        \includegraphics[width=\textwidth]{figures/limits/VH/limit_2018_VH.pdf}
        \caption{\VH --- 2018}
    \end{subfigure}
    \hfill
    \begin{subfigure}[b]{0.45\textwidth}
        \includegraphics[width=\textwidth]{figures/limits/VH/limit_Run2_VH.pdf}
        \caption{\VH --- Run-2}
    \end{subfigure}
    \caption[Observed and expected 95\,\% CL upper limits on the Higgs boson to invisible state branching fraction in the \VH category, for both the individual subcategories, and the combination of them, for each data-taking year in Run-2]{Observed and expected 95\,\% CL upper limits on the Higgs boson to invisible state branching fraction in the \VH category, for both the individual subcategories, and the combination of them, for each data-taking year in Run-2.}
    \label{fig:htoinv_limit_VH}
\end{figure}

\clearpage


%=========================================================


\section{Analysis of the \texorpdfstring{\ggH}{ggH} mode}
\label{sec:htoinv_analysis_ggF}

As with the \VH category, all of the \glspl{CR} are involved in estimating the non-multijet backgrounds in the signal region of the \ggH category. Equivalently to \ttH, the \acrshort{qcd} presence is derived using the tight and loose double sidebands. The category fractions \catFraction are calculated by the former, while the remaining terms in Eq.~\ref{eq:qcd_prediction} are computed from the latter.

Distributions of the pre-fit and post-fit yields for 2016, 2017, and 2018 are displayed in Fig.~\ref{fig:htoinv_mountain_range_ggH_SR}. Corresponding figures for the \glspl{CR} are given in App.~\ref{sec:pre_post_fit_plots_ggF_CRs}.\footnote{Add at least a paragraph discussing the pre-fit vs post-fit yields, i.e., how well the fit is able to model the background/allow it to fill any pre-fit data excesses, once I have the final results.}

\begin{figure}[htbp]
    \centering
    \begin{subfigure}[b]{0.9\textwidth}
        \includegraphics[width=\textwidth]{figures/mountain_ranges/2016/ggF/SR_tree_fit_s-abs_values_ggF_cats.pdf}
        \caption{\ggH --- 2016}
    \end{subfigure}

    \begin{subfigure}[b]{0.9\textwidth}
        \includegraphics[width=\textwidth]{figures/mountain_ranges/2017/ggF/SR_tree_fit_s-abs_values_ggF_cats.pdf}
        \caption{\ggH --- 2017}
    \end{subfigure}

    \begin{subfigure}[b]{0.9\textwidth}
        \includegraphics[width=\textwidth]{figures/mountain_ranges/2018/ggF/SR_tree_fit_s-abs_values_ggF_cats.pdf}
        \caption{\ggH --- 2018}
    \end{subfigure}
    \caption[Pre-fit and post-fit yields in the signal region for each \ggH subcategory and \ptmiss bin in each year of Run-2]{Pre-fit and post-fit yields in the signal region for each \ggH subcategory and \ptmiss bin in each year of Run-2.}
    \label{fig:htoinv_mountain_range_ggH_SR}
\end{figure}

Expected and observed limits on the signal strength parameter are presented in Fig.~\ref{fig:htoinv_limit_ggF} for each \ggH subcategory and their combination in each data taking year as well as over the full Run-2 dataset.\footnote{Add at least a paragraph discussing the limits, and their consistency, once I have the final results. Indicate problems and stuff. Relate limits to the mountain ranges.}

\begin{figure}[htbp]
    \centering
    \begin{subfigure}[b]{0.45\textwidth}
        \includegraphics[width=\textwidth]{figures/limits/ggF/limit_2016_ggF.pdf}
        \caption{\ggH --- 2016}
    \end{subfigure}
    \hfill
    \begin{subfigure}[b]{0.45\textwidth}
        \includegraphics[width=\textwidth]{figures/limits/ggF/limit_2017_ggF.pdf}
        \caption{\ggH --- 2017}
    \end{subfigure}

    \begin{subfigure}[b]{0.45\textwidth}
        \includegraphics[width=\textwidth]{figures/limits/ggF/limit_2018_ggF.pdf}
        \caption{\ggH --- 2018}
    \end{subfigure}
    \hfill
    \begin{subfigure}[b]{0.45\textwidth}
        \includegraphics[width=\textwidth]{figures/limits/ggF/limit_Run2_ggF.pdf}
        \caption{\ggH --- Run-2}
    \end{subfigure}
    \caption[Observed and expected 95\,\% CL upper limits on the Higgs boson to invisible state branching fraction in the \ggH category, for both the individual subcategories, and the combination of them, for each data-taking year in Run-2]{Observed and expected 95\,\% CL upper limits on the Higgs boson to invisible state branching fraction in the \ggH category, for both the individual subcategories, and the combination of them, for each data-taking year in Run-2.}
    \label{fig:htoinv_limit_ggF}
\end{figure}

\clearpage


%=========================================================


\section{Combined results}
\label{sec:htoinv_combined_results}

% Show the results combined over all production modes (one plot for each year), then the full combination for Run-2, ideally with VBF results as well

Upper limits for $\BRof{\higgstoinv}$ by combining all categories for a given year are given for 2016, 2017, and 2018 in Figs.~\ref{fig:htoinv_limit_likelihood_2016}, \ref{fig:htoinv_limit_likelihood_2017}, \ref{fig:htoinv_limit_likelihood_2018}, respectively. For the full Run-2 dataset, they are broken down by data taking year in Fig.~\ref{fig:htoinv_limit_likelihood_Run2_per_year} and by category in Fig.~\ref{fig:htoinv_limit_likelihood_Run2_per_cat}. Profile likelihood ratios as a function of $\BRof{\higgstoinv}$ are also presented opposite the limits.\footnote{Add some reasonable discussion of the limits and likelihoods, consistency across categories/years, once I have the final results.}

\begin{figure}[htbp]
    \centering
    \begin{subfigure}[t]{0.45\textwidth}  % top align since figures are same dimensions, but x-axis labels are larger for likelihood
        \includegraphics[width=\textwidth]{figures/limits/per_year/limit_2016_comb.pdf}
        \caption{Limit --- 2016}
    \end{subfigure}
    \hspace{0.05\textwidth}
    \begin{subfigure}[t]{0.45\textwidth}
        \includegraphics[width=\textwidth]{figures/likelihood_scan/profile_likelihood_scan_2016.pdf}
        \caption{Profile likelihood --- 2016}
    \end{subfigure}
    \caption[Observed and expected 95\,\% CL upper limit on the Higgs boson to invisible state branching fraction $\BRof{\higgstoinv}$ and the corresponding profile likelihood ratio as a function of it, for both the individual categories that target a specific production mode, as well as the combination of them, for the 2016 dataset]{Observed and expected 95\,\% CL upper limit on the Higgs boson to invisible state branching fraction $\BRof{\higgstoinv}$ (left) and the corresponding profile likelihood ratio as a function of it (right), for both the individual categories that target a specific production mode, as well as the combination of them, for the 2016 dataset. The \acrlong{sm} Higgs boson with its associated mass and production cross section are assumed.}
    \label{fig:htoinv_limit_likelihood_2016}
\end{figure}

\begin{figure}[htbp]
    \centering
    \begin{subfigure}[t]{0.45\textwidth}
        \includegraphics[width=\textwidth]{figures/limits/per_year/limit_2017_comb.pdf}
        \caption{Limit --- 2017}
    \end{subfigure}
    \hspace{0.05\textwidth}
    \begin{subfigure}[t]{0.45\textwidth}
        \includegraphics[width=\textwidth]{figures/likelihood_scan/profile_likelihood_scan_2017.pdf}
        \caption{Profile likelihood --- 2017}
    \end{subfigure}
    \caption[Observed and expected 95\,\% CL upper limit on the Higgs boson to invisible state branching fraction $\BRof{\higgstoinv}$ and the corresponding profile likelihood ratio as a function of it, for both the individual categories that target a specific production mode, as well as the combination of them, for the 2017 dataset]{Observed and expected 95\,\% CL upper limit on the Higgs boson to invisible state branching fraction $\BRof{\higgstoinv}$ (left) and the corresponding profile likelihood ratio as a function of it (right), for both the individual categories that target a specific production mode, as well as the combination of them, for the 2017 dataset. The \acrlong{sm} Higgs boson with its associated mass and production cross section are assumed.}
    \label{fig:htoinv_limit_likelihood_2017}
\end{figure}

\begin{figure}[htbp]
    \centering
    \begin{subfigure}[t]{0.45\textwidth}
        \includegraphics[width=\textwidth]{figures/limits/per_year/limit_2018_comb.pdf}
        \caption{Limit --- 2018}
    \end{subfigure}
    \hspace{0.05\textwidth}
    \begin{subfigure}[t]{0.45\textwidth}
        \includegraphics[width=\textwidth]{figures/likelihood_scan/profile_likelihood_scan_2018.pdf}
        \caption{Profile likelihood --- 2018}
    \end{subfigure}
    \caption[Observed and expected 95\,\% CL upper limit on the Higgs boson to invisible state branching fraction $\BRof{\higgstoinv}$ and the corresponding profile likelihood ratio as a function of it, for both the individual categories that target a specific production mode, as well as the combination of them, for the 2018 dataset]{Observed and expected 95\,\% CL upper limit on the Higgs boson to invisible state branching fraction $\BRof{\higgstoinv}$ (left) and the corresponding profile likelihood ratio as a function of it (right), for both the individual categories that target a specific production mode, as well as the combination of them, for the 2018 dataset. The \acrlong{sm} Higgs boson with its associated mass and production cross section are assumed.}
    \label{fig:htoinv_limit_likelihood_2018}
\end{figure}

\begin{figure}[htbp]
    \centering
    \begin{subfigure}[t]{0.45\textwidth}
        \includegraphics[width=\textwidth]{figures/limits/full_Run2/limit_Run2_comb_per_year.pdf}
        \caption{Limit --- Run-2}
    \end{subfigure}
    \hspace{0.05\textwidth}
    \begin{subfigure}[t]{0.45\textwidth}
        \includegraphics[width=\textwidth]{figures/likelihood_scan/profile_likelihood_scan_Run2_per_year.pdf}
        \caption{Profile likelihood --- Run-2}
    \end{subfigure}
    \caption[Observed and expected 95\,\% CL upper limit on the Higgs boson to invisible state branching fraction $\BRof{\higgstoinv}$ and the corresponding profile likelihood ratio as a function of it, for both the individual data taking years, as well as the combination of them, for the full Run-2 dataset]{Observed and expected 95\,\% CL upper limit on the Higgs boson to invisible state branching fraction $\BRof{\higgstoinv}$ (left) and the corresponding profile likelihood ratio as a function of it (right), for both the individual data taking years, as well as the combination of them, for the full Run-2 dataset. The \acrlong{sm} Higgs boson with its associated mass and production cross section are assumed.}
    \label{fig:htoinv_limit_likelihood_Run2_per_year}
\end{figure}

\begin{figure}[htbp]
    \centering
    \begin{subfigure}[b]{0.45\textwidth}  % top align since axis labels are larger for likelihood
        \includegraphics[width=\textwidth]{figures/limits/full_Run2/limit_Run2_comb_per_cat.pdf}
        \caption{Limit --- Run-2}
    \end{subfigure}
    \hspace{0.05\textwidth}
    \begin{subfigure}[b]{0.45\textwidth}
        \includegraphics[width=\textwidth]{figures/likelihood_scan/profile_likelihood_scan_Run2_per_cat.pdf}
        \caption{Profile likelihood --- Run-2}
    \end{subfigure}
    \caption[Observed and expected 95\,\% CL upper limit on the Higgs boson to invisible state branching fraction $\BRof{\higgstoinv}$ and the corresponding profile likelihood ratio as a function of it, for both the individual categories, as well as the combination of them, for the full Run-2 dataset]{Observed and expected 95\,\% CL upper limit on the Higgs boson to invisible state branching fraction $\BRof{\higgstoinv}$ (left) and the corresponding profile likelihood ratio as a function of it (right), for both the individual categories, as well as the combination of them, for the full Run-2 dataset. The \acrlong{sm} Higgs boson with its associated mass and production cross section are assumed.}
    \label{fig:htoinv_limit_likelihood_Run2_per_cat}
\end{figure}

% Expected limits and likelihoods only, for Scenario 5. All limit and likelihood plots from 13th August

\begin{table}[htbp]
    \centering
    \begin{tabular}{ccccc}
        \hline\hline
        Dataset & \ttH & \VH & \ggH & Combined\\\hline
        \multirow{2}{*}{2016} & X (obs.) & X (obs.) & X (obs.) & X (obs.) \\
        & 69\,\% (exp.) & 43\,\% (exp.) & 48\,\% (exp.) & 29\,\% (exp.) \\\hline
        \multirow{2}{*}{2017} & X (obs.) & X (obs.) & X (obs.) & X (obs.) \\
        & 65\,\% (exp.) & 40\,\% (exp.) & 53\,\% (exp.) & 29\,\% (exp.) \\\hline
        \multirow{2}{*}{2018} & X (obs.) & X (obs.) & X (obs.) & X (obs.) \\
        & 62\,\% (exp.) & 36\,\% (exp.) & 34\,\% (exp.) & 23\,\% (exp.) \\\hline
        \multirow{2}{*}{Run-2} & X (obs.) & X (obs.) & X (obs.) & \textbf{X (obs.)} \\
        & 40\,\% (exp.) & 24\,\% (exp.) & 27\,\% (exp.) & \textbf{16\,\% (exp.)} \\\hline\hline
    \end{tabular}
    \caption{Observed and expected upper limits on $\BRof{\higgstoinv}$ for each category and dataset in the analysis.}
    \label{tab:hinv_limits}
\end{table}
