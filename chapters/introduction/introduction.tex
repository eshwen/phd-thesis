%
% File: introduction.tex
% Author: Eshwen Bhal
% Description: Introductory chapter.
%
\chapter{Introduction}  % Alternative title: A history of dark matter observation and searches
\label{chap:intro}

% Hard to describe dark matter as an object. Do I call it a 'substance', 'thing', 'mass'? Can refer to it as 'non-luminous' if struggling for other adjectives

\epigraph{\hfill Wonder is the beginning of wisdom.}{--- Socrates}  % use hfill since the quote is short and otherwise won't flush to the right

\initial{T}he universe, in all its vastness, structure, natural laws and chaos, is comprised of only three principal components: visible matter, the ingredients of stars, planets and life, is the only one we interact with on a regular basis; dark energy, a force or manifestation of something even more mysterious, responsible for the accelerating expansion of the universe, is almost entirely unknown; and dark matter, a substance invisible in all sense of the word, that binds galaxies together and influences large scale structure in the cosmos, is the focus of this thesis.

The earliest evidence for a large, non-luminous component of the galaxy stretches back to the 1920s when Jacobus Kapteyn attempted to explain the motion of stars in the Milky Way~\cite{1922ApJ....55..302K}. Since then, a wealth of independent astrophysical observations have reinforced the existence of this aggregation not just in our own, but in countless other galaxies and cosmological bodies. Further corroborating evidence is presented in Chpt.~\ref{subsec:theory_dm_evidence}. Despite many observations, relatively little is truly known about dark matter; from its origins and its place among the particles of the \acrlong{sm}, to the reason its total mass dwarfs visible matter both in our galaxy and the universe as a whole. Given its apathy toward interacting with light---the primary medium by which we observe the dynamics of the cosmos---and the other forces that particle physics leverages to probe its foundations, the puzzling nature of dark matter is a tantalizing lure for any physicist.

While observational evidence has so far lain with astrophysics, a theoretical description and discovery of dark matter may fall into the realm of particle physics with the numerous, novel experimental searches underway. Since dark matter may be produced in measurable quantities by a particle accelerator, it is natural to utilise data from high energy collisions to conduct searches.

This thesis focuses predominantly on a search for invisibly decaying Higgs bosons in the \ttH, and \VH channels at the \acrshort{lhc}. In the \acrlong{sm} of particle physics, the only avenue by which the Higgs boson can decay invisibly is $\HepProcess{\PH \to \PZ\PZ \to \text{4}\nu}$ with a branching ratio of $\order{\text{0.1\,\%}}$~\cite{Heinemeyer:1559921}. The leading experimental upper limits on this measurement are 19\,\% from \acrshort{cms}~\cite{Sirunyan:2018owy} and 11\,\% from \acrshort{atlas}~\cite{ATLAS:2020kdi}, far higher than the predicted value. The \acrshort{vbf} channel gives the greatest contribution to the sensitivity. If undiscovered invisible particles---such as dark matter---couple to the Higgs field the branching ratio will be enhanced. As the Higgs field bestows mass to the other known elementary particles, it is possible that dark matter obtains its mass from the same mechanism. A considerably large enhancement to the invisible state branching ratio may allow for this process to be observed at the \acrshort{lhc}. At the very least, a more accurate constraint is able to exclude some models of dark matter in which the Higgs boson acts as a portal between the visible and dark sectors, such as those described in Refs.~\citenum{Djouadi:2012zc}and \citenum{KAKIZAKI201544}.

Each of the above production modes of the Higgs boson yields distinct signatures in a particle detector, with hadronic final states investigated in this thesis given their large cross sections. If the Higgs decays invisibly, the hadronic activity would be accompanied by large ``missing'' transverse momentum (explained further in Chpt.~\ref{subsec:objects_met}), a quantity that represents the momenta of particles invisible to the detector. The unique topologies of the production mechanisms can be characterised by the multiplicities of light flavour, \Pqb-tagged, and large radius \glspl{jet}, all of whose mass and substructure is compatible with hadronic decays of vector bosons and top quarks, and produced in conjunction with large missing transverse momentum. \ttH presents a particularly [challenging] search with the smallest cross section and final states with man \glspl{jet}. With the full Run-2 dataset of 137\fbinv collected by the \acrshort{cms} experiment at the \acrshort{lhc}, there is great potential to constrain the upper limit on the Higgs boson to invisible state branching fraction, and as a consequence various Higgs portal models.

In addition to Higgs physics, a generator study is presented on a new physics model resulting in \glspl{svj}: a hadronic final state in which dark matter particles are interspersed within \glspl{jet}~\cite{Cohen:2015toa,Cohen:2017pzm}. It is a novel and experimentally challenging signature easily mistaken for mismeasured \acrshort{qcd} multijet processes. Understanding how the expected signal would manifest in \acrshort{cms} is an important part of constructing an analysis to best identify it.

There is significant motivation to study dark matter from a wider, as well as a more personal, viewpoint. It is important to understand how the universe operates, and dark matter opens up the potential for new physics that improves our understanding of nature. My personal interests include the blend of particle physics and astrophysics, the opportunity to discover, and add to humanity's collective wisdom. An outline of this thesis is given as follows. A theoretical foundation is laid out in Chpt.~\ref{chap:theory} for the characteristics of the expected signal from invisibly decaying Higgs bosons and \glspl{svj}; built on top of the \acrlong{sm} which is recapitulated. The landscape of evidence, possible descriptions, and searches for dark matter are also summarised to provide context for the motivations of this thesis. Chpt.~\ref{chap:detector} consists of the predominant aspects of the design and capabilities of the \acrlong{lhc}, and the detector composition and data acquisition at the \acrshort{cms} experiment. Physics objects and quantities are defined in Chpt.~\ref{chap:objects} that are subsequently utilised in the analysis chapters: signal simulation studies of \glspl{svj} are presented in Chpt.~\ref{chap:svj}, and the search for invisibly decaying Higgs bosons is described in Chpt.~\ref{chap:higgstoinv}. Finally, a summary of this thesis and conclusions drawn from the obtained results form Chpt.~\ref{chap:conclusions}.
