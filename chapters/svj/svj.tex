\chapter{Search for dark matter through the production of semi-visible jets}
\label{chap:svj}

\initial{T}his is the analysis chapter on \glspl{svj}.

%=======
\begin{easylist}[itemize]
    \easylistprops
    & Discuss how the theoretical aspects from the Theory chapter translate into an experimental search.
    & Include object definitions, triggers, overall analysis strategy, signal production, event selection, background estimation and results/limit (including comparisons to similar searches -- monojet/dijet exotic searches). Go through everything as more of a summary/overview rather than describing everything in as much detail as Higgs to invisible.
    & Emphasise my contributions: $s$- and \tchannel signal model production and understanding. Angular variable study for \acrshort{qcd} background rejection (if used).
    & Current material: no public plots as of yet. Hope to finish \schannel analysis soon (see previous section for caveats regarding inclusion), no timeline on \tchannel or boosted \PZprime analysis.
\end{easylist}

% Can pull from Section 35 of my lab book, and all the talks I and other people from the team have given (Presentations and talks/ folder, also Other peoples/ subdirectory). Can also pull from AN for theory, translation of some theory stuff into experiment, and analysis strategy

\section{Analysis overview}
\label{sec:svj_overview}


\section{Data and simulation}
\label{sec:svj_data_sim}


\subsection{Generating signal samples in \texorpdfstring{\PYTHIA}{Pythia}}
\label{subsec:svj_signal_pythia}


\subsection{Generating signal samples in \texorpdfstring{\MADGRAPH}{MadGraph}}
\label{subsec:svj_signal_madgraph}

% Generating s-channel samples from my repo should be pretty consistent with Kevin's. But if making t-channel plots, make sure I incorporate the fixes Kevin has in his repository before running the samples.


\subsection{Triggers}
\label{sec:svj_triggers}


%\section{Tagging semi-visible jets with a boosted decision tree}
%\label{sec:svj_bdt}
% Not sure whether to include as it's not my work. Maybe a small section describing it as it's integral to the analysis

\section{Background estimation}
\label{sec:svj_background_est}
