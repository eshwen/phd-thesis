\let\textcircled=\pgftextcircled
\chapter{Theory}
\label{chap:theory}

\initial{T}his is the theory chapter.

%=======
\begin{easylist}[itemize]
\ListProperties(Style*=-- , FinalMark={)}, Margin=0.5cm)
& Give an overview of the fundamental forces and particles.
& Discuss the Standard Model in detail, emphasising certain aspects as they relate to dark matter and the Higgs field (and boson).
& Discuss dark matter: motivation, evidence for its existence (and why it can't be neutrinos/dead stars/interstellar debris, etc.), detection methods and how we can probe it at the LHC (production).
& Discuss the theory behind combined Higgs to inv.: only \acrshort{sm} process in which Higgs decays invisibly is $\PH \rightarrow \PZ\PZ \rightarrow 4\nu$ with branching ratio of 0.1\,\%~\cite{Heinemeyer:1559921}, whilst the current observed experimental limit is 19\,\% from CMS~\cite{Sirunyan:2018owy} and 26\,\% from \acrshort{atlas}~\cite{Aaboud:2019rtt}. If new, invisible particles couple to Higgs, branching ratio will be enhanced. Constraining \BR can also exclude some dark matter models.
& Discuss the theory behind the semi-visible jets analysis (main sources from Refs.~\cite{Cohen:2015toa,Cohen:2017pzm}): strongly interacting dark sector in Hidden Valley scenario with a portal to the visible sector. Mentioning dark quarks, dark confinement scale, dark hadronisation and decay, running coupling, etc.
& Explain some of the phenomenological/experimental event characteristics that overlap with both analyses, i.e., what a jet is, and maybe energy sums like \ptmiss, \HT, \htmiss, etc.
\end{easylist}


\section{The Standard Model}
\label{sec:standardmodel}


\section{Limitations of the Standard Model}
\label{sec:sm_limitations}


\section{Dark matter}
\label{sec:dark_matter}

\subsection{Measuring the branching ratio for the invisible decays of the Higgs boson}
\label{subsec:theory_higgs_to_inv}

\subsection{Searches for semi-visible jets}
\label{subsec:theory_svj}
