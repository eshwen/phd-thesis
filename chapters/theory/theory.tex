\chapter{The standard model and beyond}
\label{chap:theory}

\epigraph{It doesn’t stop being magic just because you know how it works.}{--- Terry Pratchett}

\initial{T}his thesis is comprised of experimental searches for dark matter and new physics. Chpt.~\ref{chap:higgstoinv} delves deeply into the predominantly-featured search for an invisibly decaying Higgs boson. A small chapter is dedicated to the pursuit of \glspl{svj} with Chpt.~\ref{chap:svj}. Before visiting either of which, the theoretical and phenomenological motivations must be understood to corroborate the incentives for searches at the \acrlong{lhc}. In this chapter, a brief recap of the \acrlong{sm}---with emphasis on the Higgs mechanism---will be presented along with its shortcomings, foremost the lack of a dark matter candidate. Evidence and the characteristics of dark matter are then discussed, underpinning theoretical descriptions that best fit the relic density and astrophysical observations. A short summary is also given of the state of dark matter searches at the \acrshort{lhc}. Specific interpretations in the forms of \glspl{svj} and invisibly decaying Higgs bosons are lastly examined that provide the background for the respective analysis chapters.


%=========================================================


\section{The standard model of particle physics}
\label{sec:theory_standardmodel}

The \acrlong{sm} of particle physics is the best description of nature the human race has to offer. Three of the four fundamental forces are encapsulated by it: the strong (nuclear) force, the weak (nuclear) force, and electromagnetism. The latter two may instead be considered components of a single \emph{electroweak} force~\cite{GLASHOW1961579,PhysRevLett.19.1264,Salam:1968rm}, unified above an energy of $\order{\text{100}\GeV}$; the \acrshort{lhc}, with collision energies far above this threshold, observes much electroweak physics. All of the elementary particles---the quarks, leptons, gauge bosons, and the Higgs boson---and their interactions with each other are contained within the \acrlong{sm}. These are described initially in Chpt.~\ref{subsec:theory_sm_particles}, followed by a recap of the role symmetries and gauge invariance play in their interactions with Chpt.~\ref{subsec:theory_symmetries}. Then, in Chpt.~\ref{subsec:theory_higgs_mechanism}, the Higgs mechanism and its eponymous boson, and how they factor in to the decomposition of the electroweak force into the electromagnetic and weak interactions are explained. In the aforementioned passage and in Chpt.~\ref{subsec:theory_fermion_masses}, the acquisition of mass by the electroweak gauge bosons and fermions are respectively described.


%=========================================================


\subsection{Particles of the standard model}
\label{subsec:theory_sm_particles}

The \acrlong{sm} contains a relatively small, but diverse ``zoo'' of particles. They can be divided into two distinct categories based on their internal property \emph{spin}: fermions with half-integer spin, comprise the quarks and leptons that constitute matter; and bosons with integer spin, mediate the interactions (i.e., the forces) between the fermions.

Six types, or \emph{flavours}, of quark and six flavours of lepton exist, arranged in three generations. Particles between generations share many similarities, the primary differentiator being mass. In the quark sector, the first generation contains the up \Pup and down \Pdown, the second the charm \Pcharm and strange \Pstrange, and the third the top \Ptop and bottom \Pbottom. The former particle in each generation carries an electric charge of $+\text{2/3}$ of the elementary charge $\Pe$ while the latter a value of $-\text{1/3}$. All quarks also carry colour charge, allowing them to combine into colour-neutral composite \emph{hadron} particles such as protons and neutrons. Quarks can interact via all of the forces in the \acrlong{sm}.

In the lepton sector, each generation consists of a massive particle with electric charge $-\text{1}\Pe$, and an associated electrically neutral, and extremely light, neutrino. The three generations consist of the electron \Pe and electron neutrino \Pnue, the muon \Pmu and muon neutrino \Pnum, and the tau \Ptau with the tau neutrino \Pnut. Charged leptons can interact via electromagnetism and the weak force. Though the neutrinos, lacking any electric charge, only interact via the weak force. The main properties of the fermions are summarised in Tab.~\ref{tab:fermions}.

\begin{table}[htbp]
    \centering
    \begin{tabular}{cclccr}
        \toprule
        Type & Generation & Particle & Spin & Electric charge & Mass \\ \midrule
        \multirow{6}{*}{Quark} & \multirow{2}{*}{1} & Up (\Pup) & 1/2 & $+$2/3\,\Pe & 2.2$^{+\text{0.5}}_{-\text{0.4}} \MeVcc$ \\
        & & Down (\Pdown) & 1/2 & $-$1/3\,\Pe & 4.7$^{+\text{0.5}}_{-\text{0.3}} \MeVcc$ \\
        & \multirow{2}{*}{2} & Charm (\Pcharm) & 1/2 & $+$2/3\,\Pe & 1.28$^{+\text{0.03}}_{-\text{0.04}} \GeVcc$ \\
        & & Strange (\Pstrange) & 1/2 & $-$1/3\,\Pe & 95$^{+\text{9}}_{-\text{3}} \MeVcc$ \\
        & \multirow{2}{*}{3} & Top (\Ptop) & 1/2 & $+$2/3\,\Pe & 173$\pm \text{0.4} \GeVcc$\\
        & & Bottom (\Pbottom) & 1/2 & $-$1/3\,\Pe & 4.18$^{+\text{0.04}}_{-\text{0.03}} \GeVcc$ \\
        \midrule
        \multirow{6}{*}{Lepton} & \multirow{2}{*}{1} & Electron (\Pe) & 1/2 & $-$1\,\Pe & 0.511\MeVcc \\
        & & Electron neutrino (\Pnue) & 1/2 & 0 & $< \text{0.2\eVcc}$ \\
        & \multirow{2}{*}{2} & Muon (\Pmu) & 1/2 & $-$1\,\Pe & 106\MeVcc \\
        & & Muon neutrino (\Pnum) & 1/2 & 0 & $< \text{0.2\eVcc}$ \\
        & \multirow{2}{*}{3} & Tau (\Ptau) & 1/2 & $-$1\,\Pe & 1.777\GeVcc \\
        & & Tau neutrino (\Pnut) & 1/2 & 0 & $< \text{0.2\eVcc}$ \\

        \bottomrule
    \end{tabular}
    \caption[A summary of the fermionic particles of the standard model]{A summary of the fermionic particles of the \acrlong{sm}. Masses obtained from Ref.~\citenum{PhysRevD.98.030001}.}
    \label{tab:fermions}
\end{table}

Each force is carried by one or more spin-1 gauge boson, acting to mediate the fermions' interactions. Eight flavours of the massless, electrically neutral gluon \Pgluon carries the strong force, while the equally massless and chargeless photon \Pphoton mediates the electromagnetic interaction. Three massive bosons mediate the weak interaction, the charged \PWpm and neutral \PZ. A scalar (spin-0) Higgs boson \PH is the excitation of the Higgs field that acts to bestow mass to the elementary particles. All of the bosons are summarised in Tab.~\ref{tab:bosons}.

\begin{table}[htbp]
    \centering
    \begin{tabular}{clccr}
        \toprule
        Force & Particle & Spin & Electric charge & Mass \\ \midrule
        Strong & Gluon (\Pgluon) & 1 & 0 & 0 \\
        Electromagnetism & Photon (\Pphoton) & 1 & 0 & 0 \\
        Weak & \PW bosons (\PWpm) & 1 & $\pm$1\,\Pe & $\text{80.38} \pm \text{0.01}\GeVcc$ \\
        Weak & \PZ boson (\PZ) & 1 & 0 & 91.19\GeVcc \\
        --- & Higgs boson (\PH) & 0 & 0 & $\text{125.18} \pm \text{0.16}\GeVcc$ \\
        \bottomrule
    \end{tabular}
    \caption[A summary of the bosonic particles of the standard model]{A summary of the bosonic particles of the \acrlong{sm}. Masses obtained from Ref.~\citenum{PhysRevD.98.030001}.}
    \label{tab:bosons}
\end{table}


%=========================================================


\subsection{Symmetries and gauge invariance}
\label{subsec:theory_symmetries}

The \acrlong{sm} is a gauge quantum field theory: particles are characterised by excitations of  quantum fields and their interactions described by continuous gauge symmetry groups. A symmetry is a feature in a theory where a quantity is preserved under specific transformations. This is significant for understanding what kind of interactions are allowed by each force. An SU(3) gauge group represents the interactions in the strong force, and an $\text{SU(2)} \times \text{U(1)}$ gauge group for the electroweak force. Decomposition into the weak force and electromagnetism results in separate SU(2) and U(1) groups, respectively. The interactions of the \acrlong{sm} as a whole can therefore be expressed as the product $\text{SU(3)} \times \text{SU(2)} \times \text{U(1)}$.

Noether's theorem associates a continuous symmetry of a physical system, that does not affect its Lagrangian, to a conserved charge or current~\cite{Noether_1971}. A consequence of which is that an interaction represented by a particular group must conserve the charges associated with the symmetries of said group. The generators of the group\footnote{There are $N^2 - \text{1}$ generators for SU($N$) and $N^2$ for U($N$). This is what gives rise to the eight gluons of the strong force, three (\PWplus, \PWminus, and \PZ) bosons of the weak force, and the single photon of electromagnetism.} correspond to the gauge invariant fields that mediate the associated force, i.e., the gauge bosons in the \acrshort{sm}, enforcing the conservation of the charges. With the SU(3) strong force, the mediating gluons carry colour charge. The electroweak force conserves weak isospin and weak hypercharge from the SU(2) and U(1) components, respectively. Mediated by the progenitors of the $\PW$, $\PZ$, and photon, the boson fields in the SU(2) group are referred to as $\PW_1$, $\PW_2$, and $\PW_3$, while the $B$ field is the generator of the U(1) group. As the \acrshort{sm} is described by the product of these gauge groups, it includes only gauge invariant fields that preserve the Lagrangian and equations of motions under the allowed transformations or interactions.

Symmetries in the \acrlong{sm} require the gauge bosons to be massless. For electromagnetism and the strong force, this is no issue. However, the bosons mediating the weak force have been determined experimentally to be massive~\cite{Arnison:1983mk,Bagnaia:1983zx}, posing a problem. One solution is to introduce a new field that can break the symmetry of the SU(2) group without affecting the gauge invariance elsewhere in the \acrshort{sm}. This led to the introduction of the Higgs field.\footnote{There are many who deserve credit for formulating the theory. However, for concision, the corresponding paradigms will be referred to as the \emph{Higgs field} or \emph{Higgs mechanism} henceforth.}

% In the lepton sector, each generation consists of a doublet of left-handed chiral states $(\Pnu_{\mathrm{L}}, \Plepton_{\mathrm{L}})$ arising from the SU(2) weak isospin symmetry, and a singlet of right-handed chiral states $\Plepton_{\mathrm{R}}$.


%=========================================================


\subsection{Electroweak symmetry breaking and the Higgs mechanism}
\label{subsec:theory_higgs_mechanism}

As mentioned previously, the electromagnetic and weak forces are unified into the electroweak force above a certain energy threshold. In this regime, the electroweak bosons and fermions must be massless to maintain gauge invariance. The former obtain their mass by interacting with the Higgs field, the means by which is labelled the \emph{Higgs mechanism}~\cite{PhysRevLett.13.321,PhysRevLett.13.508,PhysRevLett.13.585}.

The simplest model by which the Higgs mechanism can be accommodated in the \acrshort{sm} is by introducing an SU(2) doublet of complex scalar Higgs fields:
\begin{equation} % Ensure no blank lines before or after equation so spacing and indentation is correct
    \HiggsField = \begin{pmatrix} \phi^+ \\ \phi^0 \end{pmatrix} = \frac{1}{\sqrt{2}} \begin{pmatrix} \phi_1 + i\phi_2 \\ \phi_3 + i\phi_4 \end{pmatrix}
    \label{eq:higgs_field}
\end{equation}
It possesses four degrees of freedom, i.e., one per gauge boson in the electroweak sector. The fields introduce additional terms in the \acrlong{sm} Lagrangian, most importantly a potential of the form
\begin{equation}
    V(\HiggsField) = \mu^2 \HiggsField^{\dagger} \HiggsField + \lambda (\HiggsField^{\dagger} \HiggsField)^2
    \label{eq:higgs_potential}
\end{equation}
which is the most general scalar potential that is also $\text{SU(2)} \times \text{U(1)}$ invariant. With both quadratic and quartic components, positive values of both free parameters $\mu^2$ (related to mass) and $\lambda$ (related to self-interaction) would yield a single, stable minimum for the potential. In a physical context, this corresponds to the universe pre-electroweak symmetry breaking. But by setting $\lambda > \text{0}$ and $\mu^2 < \text{0}$, what was once the minimum becomes an unstable local maximum and a new, degenerate circle of minima in the $\phi_1 - \phi_2$ plane can be found with values
\begin{equation}
    \HiggsField^{\dagger} \HiggsField \lvert_{\mathrm{min}} = -\frac{\mu^2}{2\lambda}
    \label{eq:higgs_mu_lambda}
\end{equation}
forming the familiar ``wine bottle'' potential. An illustration with a toy Higgs boson is given in Fig.~\ref{fig:higgs_potential}.

\begin{figure}[htbp]
    \centering
    \includegraphics[width=0.5\textwidth]{./figures/higgs_potential_pptx.pdf}  % Try to make this larger while still being positioned well
    \caption[A depiction of the Higgs potential $V$ as a function of the component fields]{A depiction of the Higgs potential $V$ as a function of the component fields. The Higgs boson is initially at $(\phi_1, \phi_2) = (\text{0}, \text{0})$, an unstable local maximum. It then tumbles into the minimum value of $V$ and triggers electroweak symmetry breaking.}
    \label{fig:higgs_potential}
\end{figure}

To spontaneously break the $\text{SU(2)} \times \text{U(1)}$ symmetry of the electroweak force, a non-zero vacuum expectation value $v$ must be assigned from the set of minima belonging to the potential $V$. Since the ability to select any potential is a symmetry, assigning a value breaks said symmetry. For the fields $\phi_1$, $\phi_2$, and $\phi_4$, $v$ can simply be set to zero. However, as a neutral field, $\phi_3$ must be assigned a non-zero value in order for the photon to remain massless after the symmetry is broken. It is allocated a value
\begin{equation}
    \bra{0} \mkern-1mu \phi_3 \mkern-1mu \ket{0} = v^2 = -\frac{\mu^2}{\lambda}
    \label{eq:higgs_vev}
\end{equation}
By expanding \HiggsField around the chosen vacuum and using the unitary gauge, it can be seen that all of the fields other than $\phi_3$---known as \emph{Goldstone fields} in this paradigm---are eliminated. The originally massless $\PW_{1,2,3}$ and $B$ fields then mix, with some states gain mass by ``eating'' the Goldstone bosons. The $\PW_1$ and $\PW_2$ fields transform into the $\PWpm$ bosons:
\begin{equation}
    \PWpm = \frac{1}{\sqrt{2}} (\PW_1 \mp i\PW_2)
    \label{eq:EWSB_W}
\end{equation}
While the $\PW_3$ and $B$ fields mix to produce the massless photon and massive \PZ boson:
\begin{equation}
    \begin{pmatrix} \Pphoton \\ \PZ \end{pmatrix} = \begin{pmatrix} \cos\thetaW & \sin\thetaW \\ -\sin\thetaW & \cos\thetaW \end{pmatrix} \begin{pmatrix} B \\ \PW_3 \end{pmatrix}
    \label{eq:EWSB_photon_Z}
\end{equation}
In the new U(1) symmetry group of electromagnetism (mediated by the photon), electric charge is the conserved current, manifesting as a combination of weak hypercharge and the third component of weak isospin. For the SU(2) weak interaction, weak isospin becomes the conserved quantity. Due to only left-handed chiral states of fermions (and right-handed antifermions) possessing a non-zero weak isospin, certain interactions are not allowed by the weak force. Only left-handed fermions and right-handed antifermions couple to the \PW bosons that participate in \emph{charged current} interactions. This maximally violates the parity symmetry conserved by the other forces. Left- and right-handed fermions also couple differently to the \PZ boson that mediates \emph{neutral current} interactions.

The masses of the electroweak bosons can be calculated using the vacuum expectation value, and the coupling strengths of the weak force $g$ and electromagnetism $g'$:
\begin{equation}
    \begin{aligned}
m_{\PW} &= \frac{1}{2} gv, \\[1em]
m_{\PZ} &= \frac{1}{2} (g\cos\thetaW + g'\sin\thetaW)^2 v = \frac{1}{2}\frac{gv}{\cos\thetaW}
    \end{aligned}
    \label{eq:w_z_masses}
\end{equation}
where $\thetaW$ is the \emph{weak mixing angle} or \emph{Weinberg angle}. It is a quantity that represents the degree to which the weak and electromagnetic forces mix, defined as
\begin{equation}
    \cos\thetaW = \frac{m_{\PW}}{m_{\PZ}} \ \text{or} \ \tan\thetaW = \frac{g'}{g}
    \label{eq:weak_mixing_angle}
\end{equation}
It is a free parameter in the \acrlong{sm} with its value constrained by measuring the $\PW$ and $\PZ$ boson masses. With the latest results in Ref.~\citenum{PhysRevD.98.030001}, $\thetaW = \text{28.74}^{\circ}$ and $v = \text{246}\GeVcc$. The mass of the Higgs boson is given by
\begin{equation}
    m_{\PH} = \sqrt{2\lambda v^2} = \sqrt{-2\mu^2}
    \label{eq:higgs_mass}
\end{equation}


%=========================================================


\subsection{Masses of the fermions}
\label{subsec:theory_fermion_masses}

The quarks and leptons in the \acrshort{sm} acquire their masses from the Higgs field by means of a Yukawa coupling $y_f$, which describes the interaction between a scalar (Higgs) field and Dirac (fermion) field. As a function of $y_f$ and the vacuum expectation value of the Higgs field, the fermion masses are calculated as
\begin{equation}
    m_f = \frac{1}{\sqrt{2}} y_f v
    \label{eq:fermion_mass}
\end{equation}
They are directly proportional to their coupling strength to the Higgs, and are all free parameters in the \acrshort{sm} so must be constrained by experiment. The decay width $\Gamma$ of the Higgs boson into fermions is computed as
\begin{equation}
    \Gamma(\HepProcess{\PH \to f\bar{f}}) = \frac{m_{\PH}}{8\pi} \left( \frac{m_f}{v} \right)^2 \Nc \left( 1 - \frac{4 m_f^2}{m_{\PH}^2} \right)^{\frac{3}{2}}
    \label{eq:higgs_decay_width_fermions}
\end{equation}
which is proportional to the squared mass of the decay products. The \emph{branching ratio} $\BR$---the probability of a particle decaying into a given final state---is simply the ratio of the partial width to the total width, the latter being the sum of partial widths for all the particle's decay modes.


%=========================================================


\subsection{Limitations of the standard model}
\label{subsec:sm_limitations}

% Check Lancaster and other summer school notes for other limitations, specifically referencing things that can tie into dark matter

Despite the \acrlong{sm} providing precise predictions of three of the four fundamental forces and the particles interacting through them, there are many experimental observations that it cannot currently explain. Neutrino masses, dark matter, dark energy, and gravity all escape its description. Many of the important parameters in the \acrshort{sm} do not have predicted values and so must be experimentally measured.

The Hierarchy Problem is one of the more serious issues facing the \acrlong{sm}. It may be explained in different manners that emphasize certain aspects. But inherently, it is a question of the disparity between energy scales of the fundamental forces---particularly relating to the weak force and gravity. The masses of the electroweak and Higgs bosons of $\order{\text{100}\GeVcc}$ are much smaller than than the Planck mass of $\order{\text{10}^{19}\GeVcc}$. The mass term for the Higgs boson is $\mu^2 \HiggsField^{\dagger} \HiggsField$ in the \acrshort{sm}. Invariance under a gauge or global symmetry in the Higgs field leads to the mass being open to radiative corrections up to the Planck scale. It appears that, in nature, these very large corrective terms to the Higgs boson mass cancel to give the familiar $m_{\PH} = \text{125}\GeVcc$~\cite{Chatrchyan:2012xdj,Aad:2012tfa}. It is deemed unnatural to expect cancellations to such a degree, i.e., one part in $\text{10}^{17}$. This \emph{fine-tuning} of parameters in the \acrlong{sm} is something that unified or natural theories desperately try to avoid.

Some theories \acrfull{bsm} like \acrfull{susy} provide well-motivated cancellations by introducing supersymmetric particles. In certain scenarios, some of these particles should exist at the \TeVns scale. In the \acrshort{sm}, the largest correction to the Higgs mass derives from the top quark, since its Yukawa coupling to the Higgs is the strongest. At one-loop order, new physics at the $\order{\TeVns}$ scale is required, with new particles coupling to the Higgs field to prevent these corrections from being unreasonably large~\cite{Farina:2013ssa}. Arguments such as this give credence to new physics being discoverable at particle accelerators such as the \acrlong{lhc}.


%=========================================================


\section{The dark matter landscape}
\label{sec:theory_dm_landscape}

\subsection{Evidence}
\label{subsec:theory_dm_evidence}

There is a wealth of evidence affirming the existence of dark matter in the universe. The Coma Cluster is a famous example: 90\,\% of its mass is thought to arise from dark matter, confirmed by its large mass-to-light ratio of 400 $M_{\odot} / L_{\odot}$~\cite{Yozin:2015mla}. Further evidence is that the rotation curves of most galaxies are roughly flat~\cite{1996MNRAS-281-27P}, contrary to the expected Keplerian relationship $(v_r \propto r^{-1/2})$ expected from solely visible matter. On a galactic scale, dark matter is sprinkled in a mostly spherical halo that spans beyond the observable disc. The inclusive dark matter mass increases linearly~\cite{2009arXiv0901.0632E} to compensate for the decline expressed by visible matter~\cite{1970ApJ-160-811F,1992AandA-256-19B}. Gravitational lensing is another observational tool subject to influence from dark matter. Images of galaxies and other objects captured by this method appear distorted from a large gravitational field between the source and observer warping its local spacetime~\cite{2010GReGr..42.2177H}. Arcs, ellipses and Einstein rings of smeared galaxies are often seen when dark matter is present.

\clearpage

While there are no widely-accepted estimations, it is believed that 85--95\,\% of the Milky Way is comprised of dark matter~\cite{2005MNRAS.364..433B,2006MNRAS.370.1055B,Kafle:2014xfa}. Though these approximations include non-visible identifiable matter such as dim stars, black holes and neutron stars, the term \emph{dark matter} is typically reserved for the non-luminous, \emph{non-baryonic} segment that pervades the cosmos. From the latest results of the Planck mission, the energy density of the observable universe is composed of 26.5\,\% dark matter~\cite{Aghanim:2018eyx}. This result follows the Lambda cold dark matter ($\Lambda\text{CDM}$) model to describe the constituents and evolution of the universe, which is often referred to as the cosmological analog of the \acrlong{sm}. From the calculations, postulations, and observations presented above, the following properties of dark matter can be deduced:

\begin{easylist}[itemize]
    \easylistprops
    & It is electrically neutral as it does not interact with electromagnetic radiation. Hence, the adjective ``dark'' in dark matter.
    & It is non-relativistic, or \emph{cold}. Its velocity within galaxies is similar to the inhabiting stars~\cite{Herzog-Arbeitman:2017fte,Bhattacharjee:2012xm}, since the combination of visible and dark matter drives the measured rotation curves. From classical mechanics, galactic dark matter \emph{must} be cold since a velocity above the gravitational escape velocity of the galaxy would eject high speed particles.
    & It is stable, at least on the timescale of the age of the universe. Dark matter production is postulated to have occurred only in the early universe via a thermal freeze-out mechanism. Hence, the remaining fraction has been present for a considerable time. Since most galaxies are dominated by dark matter and the gravitational influence from only the visible matter is too small to maintain itself, they could not have developed with out it. This supports the idea of bottom-up structure formation in the universe: smaller galaxies form around gravitational potential wells induced by coalescing dark matter, then merge to form larger structures~\cite{doi:10.1093-mnras-183.3.341}.
    & Its interaction with matter and itself is very weak, or even non-existent. The Bullet Cluster---an astronomical object consisting of two colliding galaxy clusters---is the best example of this inference. From measurements of, predominantly, x-ray emission and gravitational lensing, it was found that while there is a substantial amount of dark matter present, interaction with itself and the visible matter surrounding it was minimal at most~\cite{BulletClusterDMevidence}. A kinematic explanation for the spherical distribution and low velocity of dark matter in galaxies can be explained by its collisionless nature. During the formation of a galaxy or planetary system, visible matter frequently collides, dissipating angular momentum and collapsing into a disc.
\end{easylist}


%=========================================================


\subsection{Theoretical descriptions}
\label{subsec:theory_dm_descritions}

Dark matter may have been forged in the universe via one of many possible mechanisms. The most popular is described as a \emph{thermal freeze-out} process. In the hot, early universe when the thermal background allowed spontaneous pair production of particle dark matter, it was generated in abundance. During this period, the particles may also have frequently annihilated seeing that the cosmos was still small. Inevitably, the universe expanded and cooled; the temperature became too low to allow significant production~\cite{Baldes:2017gzw}. Matter was further separated and the dark matter annihilation rate decreased, leaving a behind the \emph{thermal relic} that is observed today. These remaining particles were attracted via gravity, forming filaments throughout the universe. The potential wells they induced allowed the progenitors of galaxies to form within.

Full derivations of the thermal freeze-out of dark matter can be found in literature~\cite{cosmic_abundances_stable_particles,Bender:2012gc}, with the \acrshort{wimp} Miracle as a consequence: with relatively few assumptions, the correct dark matter relic abundance can be recovered by requiring a \acrfull{wimp} with a mass of $\order{\text{\GeVns--\TeVns}}$ dependent on the annihilation cross section, and an interaction strength similar to that of the weak force. This mass range is accessible at contemporary colliders such as the \acrshort{lhc}, and perhaps coincidentally, around the electroweak energy scale. It is common for figures that depict the \acrshort{wimp} dark matter density over time to plot the yield $n_{\chi}/s$ as a function of the dimensionless parameter $x = m_{\chi}/T$. In the former variable, $n_{\chi}$ is the number density and $s$ is the entropy density. In the latter, $m_{\chi}$ is the dark matter mass and $T$ is the average temperature of the universe, which serves as a measure of its age due to the temperature decreasing over time. An example is given in Fig.~\ref{fig:theory_dm_abundance}.

\begin{figure}[htbp]
    \centering
    \includegraphics[width=0.75\textwidth]{figures/dm_abundance.pdf}
    \caption[A measure of the comoving number density of WIMP dark matter as a function of time with projections for several particle masses]{A measure of the comoving number density of \acrshort{wimp} dark matter as a function of time with projections for several particle masses. The black curve represents the scenario in which dark matter remains in equilibrium with the \acrlong{sm}. Figure taken from Ref.~\citenum{Han:2013gba}.}
    \label{fig:theory_dm_abundance}
\end{figure}

The time of the dark matter freeze-out epoch is somewhat insensitive to the mass and annihilation cross section. Approximate solutions to the Boltzmann equation for a time-dependent $n_{\chi}$---where dark matter is modelled as a weakly-interacting, diffuse gas of particles---suggest $x_f \sim \text{20}$~\cite{Lisanti:2016jxe,Bender:2012gc}. Stronger dark matter interaction leads to decoupling at a later time and a lower number density. The approximate value of $x_f$ is significant in that it supports the electroweak-scale mass of \acrshort{wimp}s. A higher mass must be balanced by a larger annihilation cross section to achieve the correct relic density, to which it tends asymptotically from the point of decoupling.

Another popular mechanism, targeting low-mass dark matter, is the \emph{freeze-in} process~\cite{Hall:2009bx,Krnjaic:2017tio}. In this postulate, dark matter is not produced thermally in the early universe. Instead, it emerges through interactions between \acrshort{sm} particles such as collisions, or decays of those heavier than dark matter. The comoving density increases with time until it plateaus from of the cooling of the universe, where \acrshort{sm} particles are generally stable enough and too low energy to produce dark matter in any meaningful quantity. The relic abundance can therefore be reclaimed from a combination of the initial thermal distributions, the dark matter mass, and the interaction strength, similar to the freeze-out process. In order to obey cosmological observations, particularly the fact that it is cold, the masses expected for freeze-in dark matter particles are of $\order{\keVns}$ or heavier.


%=========================================================


\subsection{Searches}
\label{subsec:dm_searches_lhc}

A coalescence of observation from astrophysics and application from particle physics has paved the way for a nimiety of dark matter models that can be tested by either discipline. Searches dark matter can be classified into three distinct methods with unique signatures (paired with a visual summary in Fig.~\ref{fig:dm_detection_methods}):

\begin{easylist}[itemize]
    \easylistprops
    & \textbf{Direct}: dark matter may interact with visible matter on small scales, scattering from \acrlong{sm} particles~\cite{Schumann:2019eaa}. The recoil the \acrshort{sm} particles experience could be detected by highly-sensitive, low background experiments such as \acrfull{lz}~\cite{Akerib:2019fml} which specialises in the search for \acrshort{wimp} dark matter within a wide range of masses.
    & \textbf{Indirect}: if dark matter interacts with itself, it may annihilate to produce showers of high energy photons or pions. Background estimation is difficult since the signatures can be highly model-dependent. The particles may be of a continuum---from hadronisation and radiation of the decay products---or contain features, such as internal radiation from the propagator in the interaction or from loop-level processes~\cite{Conrad:2017pms}. Large ranges of the annihilation cross section and dark matter mass can be probed with telescopes already searching for these characteristic events.
    & \textbf{Production}: dark matter may have been abundantly produced in the hot, early universe. High energy particle accelerators such as the \acrshort{lhc} can reproduce these conditions, with the \acrshort{wimp} Miracle reinforcing the idea that dark matter may exist in these accessible mass ranges. Many \acrshort{bsm} theories accommodate dark matter candidates with a diverse spectrum of final states that can be investigated by analysing \acrshort{lhc} data.
\end{easylist}

\begin{figure}[htbp]
    \centering
    \includegraphics[width=0.7\textwidth]{figures/DM_detection_methods.png}
    \caption[A visual representation of the three main types of dark matter detection: direct, indirect, and production]{A visual representation of the three main types of dark matter detection: direct (dark matter recoiling from \acrlong{sm} particles); indirect (annihilation of dark matter); and production (dark matter created in high energy particle collisions).}
    \label{fig:dm_detection_methods}
\end{figure}

Since detection at the \acrshort{cms} experiment from the production mechanism is the subject of this thesis, it is important to establish the current state of dark matter searches at the \acrshort{lhc}, the world's most powerful particle accelerator that provides the infrastructure for \acrshort{cms} to collect data. The \acrshort{lhc} principally collides protons at centre of mass energies up to \comruntwo. These exceptionally high energies allow the conditions in the very early universe to be simulated in which heavy, unstable particles were produced plentifully. As a result, many theories can be investigated that predict heavy particles that do not exist in the universe today. Some of these, such as \acrshort{susy}~\cite{Martin:1997ns}, sterile neutrinos~\cite{doi:10.1142/S0218301313300191}, and Kaluza-Klein states~\cite{Han:1998sg} contain dark matter candidates that can be specifically searched for, or indirectly inferred if a theory is experimentally proven. Despite the success of the \acrlong{sm} in explaining much of the natural world, it does not substantiate the existence of dark matter. \acrshort{bsm} theories can therefore gain traction. Fig.~\ref{fig:dm_masses_xsecs} illustrates the masses and interaction cross sections of many dark matter candidates. \acrshort{wimp}s (highlighted by the purple rectangle) are the the subject of several searches at the \acrshort{lhc} since the expected mass ranges and cross sections are accessible there.

\begin{figure}[htbp]
    \centering
    \includegraphics[width=0.6\textwidth]{figures/dm_masses_xsecs.jpg}
    \caption[The expected masses and interaction cross sections of a set of dark matter candidates]{The expected masses and interaction cross sections of a set of dark matter candidates. The \acrshort{lhc}, with its centre of mass energy of 13\TeV, is best suited to targeting \glspl{wimp}. Figure acquired from Ref.~\citenum{Conrad:2017pms}.}
    \label{fig:dm_masses_xsecs}
\end{figure}

Two avenues are usually considered when attempting to discover dark matter: explicit searches for the signatures of dark matter production, and anomalies in precision measurements. The former is quite common, with many theories and models tested at the \acrshort{lhc}'s general purpose detectors, \acrshort{atlas} and \acrshort{cms}. Searches at \acrshort{cms} have been performed for promptly-decaying and \emph{long-lived} \acrlong{susy} in hadronic final states~\cite{CMS-PAPER-SUS-15-005-published,SUS16038published}. Searches for specific supersymmetric particles in a variety of decay modes have been conducted by both experiments~\cite{CANEPA2019100033}. In many of these cases, the \acrfull{lsp} is considered to be a dark matter candidate. $R$-parity conservation is predicted (or even enforced) in many \acrshort{susy} models~\cite{Martin:1997ns}, which prevents the decay of \acrshort{lsp} and any lighter, \acrlong{sm} particles that have been observed to be stable. While \acrlong{susy} is the most popular \acrshort{bsm} theory, due in part to its numerous interpretations and approaches for discovery, many others have also been explored at the LHC. From microscopic black holes~\cite{Khachatryan:2010wx} to dark photons~\cite{dark_photons_CMS_2019}, there are extensive propositions that have the potential to uproot the \acrlong{sm}. The analyses above are usually characterised by large missing transverse momentum (\ptvecmiss). Both its magnitude and direction in relation to the visible event content are important, especially when searching for topologies like \glspl{svj}. 

Precise measurements of \acrlong{sm} parameters is the other method often consulted in the hopes of attributing discrepancies to new physics. For example, attempts to explain anomalies in the $\HepProcess{\Pqb \to \Pqs}$ transition include dark matter candidates~\cite{Vicente:2018xbv,another_b_s_anomaly_paper}. An investigation into the measurement of the Higgs boson to invisible state branching ratio is extensively detailed in Chpt.~\ref{chap:higgstoinv} along with interpretations that can accommodate dark matter. Since the same laws of physics and accepted descriptions of the universe are shared across fields, results from new measurements or searches in one sector can influence others. Notably, stronger constraints on \acrlong{sm} measurements like the above can exclude \acrshort{bsm} theories searched for in direct and indirect dark matter detection.


%=========================================================


\section{Measuring the branching ratio of invisibly decaying Higgs bosons}
\label{sec:theory_higgs_to_inv}

The Higgs boson has caught the attention of the high energy physics community, and even the public eye, like no other particle in recent memory. Its discovery in the $\HepProcess{\PH \to \Pphoton\Pphoton}$ and $\HepProcess{\PH \to \PZ\mkern-1mu\PZ \to \text{4}\Plepton}$ channels in 2012---independently by both \acrshort{cms}~\cite{Chatrchyan:2012xdj} and \acrshort{atlas}~\cite{Aad:2012tfa}---realised one of the paramount goals of the \acrshort{lhc}'s construction. The particle itself is not necessarily exciting. Rather, it confirms the existence of the Higgs \emph{field} that pervades the universe and gives mass to the elementary particles via the exchange of its eponymous boson~\cite{PhysRevLett.13.321,PhysRevLett.13.508,PhysRevLett.13.585}. Its discovery, one might think, was the end of the discussion of the Higgs boson. However, it was only the beginning.

Many observations of the Higgs, such as its predominant decay mode $\HepProcess{\PH \to \Pqb\APqb}$, were not seen until recently by \acrshort{cms}~\cite{Sirunyan:2018kst} or \acrshort{atlas}~\cite{Aaboud:2018zhk}. Constraints on its other properties have also been placed, such as its resonance width and branching ratios to several final states~\cite{PhysRevD.98.030001}. Fully understanding the Higgs boson is important to understanding the Higgs field and the wider \acrlong{sm}. Precision measurements in tension with \acrshort{sm} predictions can also be a window to new physics. Measuring the \higgstoinv branching ratio aims to do just that.

The only \acrshort{sm} process in which Higgs boson can decay invisibly is to four neutrinos via a pair of $\PZ$ bosons, possessing a branching ratio of only $\order{\text{0.1\,\%}}$ while leading experimental upper limits are $\order{\text{10\,\%}}$. If the Higgs field couples to dark matter, the overall $\BRof{\higgstoinv}$ could increase substantially with possible observations at the \acrshort{lhc}. If not, the upper limit on the branching ratio can still be chipped away, further constraining or excluding a selection of dark matter models. 

Since the other elementary particles also acquire mass from the Higgs field, the same may be true for dark matter. Higgs \emph{portal} models have been theorised that connect the visible sector of the \acrlong{sm} to a dark sector where particle dark matter resides~\cite{higgs_portal_singlet_dm,Arcadi:2019lka}. Certain models also predict a detectable presence at the \acrshort{lhc} from a sufficient production rate~\cite{Boveia:2018yeb}, perhaps even with data obtained during Run-2~\cite{Abercrombie:2015wmb}. A dark matter mass of $m_{\chi} < m_{\PH}/\text{2}$ would allow for direct pair production from the Higgs boson---a much stronger signal than via an intermediate decay, as is the case for the $\text{4}\nu$ final state.

% If models exist, mention briefly about potential theories with dark matter candidates being able to fix the hierarchy problem (present in a mathematical context if possible).

An analysis in search of this decay is provided thoroughly in Chpt.~\ref{chap:higgstoinv}. Constraints on the experimental side stem largely from the different channels in which a Higgs boson can be produced. These are outlined in Chpt.~\ref{subsec:theory_higgs_production_modes}, and must all be considered when examining such a rare process that is also difficult to distinguish amongst a large background. Previous results from searches for individual modes, including subsequent combinations, are documented in Chpt.~\ref{subsec:theory_hinv_prev_results}.


%=========================================================


\subsection{Production modes of the Higgs boson}
\label{subsec:theory_higgs_production_modes}

At the \acrshort{lhc}, the most common mechanisms for producing a Higgs boson are \acrfull{vbf}, gluon-gluon fusion (\ggF or \ggH), associated production from top quarks (\ttH), and associated production from a vector boson (\VH). Feynman diagrams of these processes are shown in Fig.~\ref{fig:higgs_feynman_diagrams}. Additional diagrams for \ggH involve a square top quark loop and/or initial state radiation. The \ZH process can be initiated by $\HepProcess{\Pgluon\Pgluon\to \ZH}$ as well as $\HepProcess{\Pp\Pp \to \ZH}$. They all have very different characteristics, production rates, and event signatures, complementing each other and allowing analyses to cover all bases with orthogonal parameter spaces to target them individually. One common feature of these final states is the presence of at least one quark. The hadronic constituents in the decay products of a collision often shower due to colour confinement, producing collimated sprays of hadrons called \emph{\glspl{jet}}. In a detector, these are represented by clusters of hadronic energy deposits. Algorithms at each stage of data acquisition (see Chpt.~\ref{subsec:cms_recording_data} for those in \acrshort{cms}) can reliably connect these back to the individual quark decays so one has some certainty of the process they are observing. The cross section of each mechanism at \comruntwo is detailed in Tab.~\ref{tab:htoinv_signal_xsecs}.

\begin{figure}[htbp]
    \centering
    \begin{subfigure}[b]{0.45\textwidth}
        \includegraphics[width=\textwidth]{figures/feynman_diagrams/ttH.pdf}
        \caption{\ttH}
    \end{subfigure}
    \hfill
    \begin{subfigure}[b]{0.45\textwidth}
        \includegraphics[width=\textwidth]{figures/feynman_diagrams/VBF.pdf}
        \caption{\acrshort{vbf}}
    \end{subfigure}
% blank line to start new row
    \begin{subfigure}[b]{0.45\textwidth}
        \includegraphics[width=\textwidth]{figures/feynman_diagrams/VH.pdf}
        \caption{\VH}
    \end{subfigure}
    \hfill
    \begin{subfigure}[b]{0.45\textwidth}
        \includegraphics[width=\textwidth]{figures/feynman_diagrams/ggF.pdf}
        \caption{\ggH}
    \end{subfigure}
\caption[A subset of the Feynman diagrams for the four predominant production mechanisms of the Higgs boson at the LHC]{A subset of the Feynman diagrams for the four predominant production mechanisms of the Higgs boson at the \acrshort{lhc}.}
\label{fig:higgs_feynman_diagrams}
\end{figure}

\begin{table}[htbp]
    \centering
    \begin{tabular}{cll}
        \toprule
        Production mode & Cross section (pb) & Accuracy \\\midrule
        \acrshort{vbf} & 3.77 & \acrshort{nnlo} \acrshort{qcd} and \acrshort{nlo} electroweak \\
        $\ttH$ & $\text{5.07} \times \text{10}^{-1}$ & \acrshort{nlo} \acrshort{qcd} and \acrshort{nlo} electroweak \\
        $\WplusH$ & $\text{8.31} \times \text{10}^{-1}$ & \acrshort{nnlo} \acrshort{qcd} and \acrshort{nlo} electroweak \\
        $\WminusH$ & $\text{5.27} \times \text{10}^{-1}$ & \acrshort{nnlo} \acrshort{qcd} and \acrshort{nlo} electroweak \\
        $\HepProcess{\pp \to \ZH}$ & $\text{8.84} \times \text{10}^{-1}$ & \acrshort{nnlo} \acrshort{qcd} and \acrshort{nlo} electroweak \\
        $\HepProcess{\Pgluon\Pgluon \to \ZH}$ & $\text{1.23} \times \text{10}^{-1}$ & \acrshort{lo} \acrshort{qcd} (\acrshort{nlo} + NLL corrections) \\
        \ggH & $\text{4.86} \times \text{10}^1$ & N3LO \acrshort{qcd} and \acrshort{nlo} electroweak \\
        \bottomrule
    \end{tabular}
    \caption[Cross sections of the Higgs boson production modes]{Cross sections of the Higgs boson production modes. They are calculated at \comruntwo at the highest orders available and obtained from Ref.~\citenum{Cepeda:2019klc}.}
    \label{tab:htoinv_signal_xsecs}
\end{table}

% Xsecs from https://twiki.cern.ch/twiki/bin/view/LHCPhysics/CERNYellowReportPageAt13TeV and https://twiki.cern.ch/twiki/bin/view/LHCPhysics/CERNHLHE2019


%=========================================================


\subsubsection{Vector boson fusion (VBF)}
\label{subsubsec:theory_hinv_VBF_mode}

A \acrshort{vbf} topology is exhibited by a \tchannel exchange of two vector bosons radiated by the incident quarks, which then combine to form a new particle such as a Higgs boson. Since the masses of the \PW and \PZ bosons are more than half the Higgs' mass, it can easily be produced on shell. The recoil of the quarks from the Higgs boson characterises the visible system: two \glspl{jet} with a large combined invariant mass, usually with a large separation in pseudorapidity $\eta$ but small in azimuthal angle.\footnote{These coordinates are described in Chpt.~\ref{subsubsec:geometry}.} The \glspl{jet} move in opposite directions, one in $+\eta$ and the other in $-\eta$, but are usually contained in the same horizontal half of the detector.


%=========================================================


\subsubsection{Associated production from top quarks (\texorpdfstring{\ttH}{ttH})}
\label{subsubsec:theory_hinv_ttH_mode}

In \ttH, a \ttbar pair is produced from the collision. A virtual top quark \Ptop and antiquark \APtop produced in association with their real counterparts annihilate to produce the Higgs boson. As it decays invisibly, it is the remaining \Ptop and \APtop in the event that lead to three classes of final state. The \Ptop quark decays almost exclusively to $\Pbottom\PWplus$ (with $\HepProcess{\APtop \to \APbottom\PWminus}$)~\cite{PhysRevD.98.030001}. In a resolved system where top quarks possess low to moderate momentum, the multitude of available \Pbottom-tagging algorithms can distinguish the decays of the \Pbottom quark. The products of the \PW boson are the determining factor of the final state. Hadronically-decaying $\mkern-2mu\PW\text{s}$ ($\HepProcess{\PW \to \Pquark\APquark}$), of course, produce pairs of \glspl{jet}. But they can also decay into a lepton and neutrino. The final states then all have \ptvecmiss and up to two \glspl{bjet} in common. Several \glspl{jet} may accompany them (the hadronic channel), or fewer \glspl{jet} with a single lepton (the \emph{semi-leptonic} channel), or simply two leptons (the \emph{dileptonic} channel). The magnitude and direction of the \ptvecmiss in the latter two channels may be affected by the neutrinos, depending on their direction and energy.

In a boosted system where the top quarks have significant \pt, it is often difficult to tag \glspl{bjet}, especially if one is searching in the hadronic channel. The decay products are not well separated and can merge into large, ``fat'' \glspl{jet}. Recently-developed algorithms can assist in this case by inspecting the substructure of these fat \glspl{jet} to classify, for example, boosted topologies originating from \Ptop quarks as well as \PVec bosons to identify \ttH events.


%=========================================================


\subsubsection{Associated production from a vector boson (\texorpdfstring{\VH}{VH})}
\label{subsubsec:theory_hinv_VH_mode}

A Higgs boson is radiated by the vector boson \PVec in the \VH mechanism. Parallels can be drawn with \ttH as the decay of the \PVec determines the search channel. Resolved and boosted systems are also possible. In the resolved case, a dijet pair with an invariant mass close to that of the parent boson would distinguish the hadronic channel. \Pbottom-taggers can be exploited if the decay is to a \Pbottom quark, i.e., $\HepProcess{\PZ \to \Pbottom\APbottom}$.\footnote{Other potential decay modes, such as $\HepProcess{\PW \to \Pbottom\Pup}$ and $\HepProcess{\PW \to \Pbottom\Pcharm}$ are suppressed in the CKM matrix, yielding small production rates at the \acrshort{lhc}.} Single lepton channels are possible for \WH and dilepton for \ZH. For a boosted \PVec, one expects the products to the collimated into a fat \gls{jet}, at least in the hadronic channel. As with \ttH, one can take advantage of novel tagging algorithms to capture these scenarios. 


%=========================================================


\subsubsection{Gluon-gluon fusion (\texorpdfstring{\ggH}{ggH})}
\label{subsubsec:theory_hinv_ggF_mode}

Despite \ggH having the largest cross section of the four modes, its upper limit on $\BRof{\higgstoinv}$ is the weakest. The Higgs boson is created through the loop-level fusion of the initial state gluons, normally mediated by a top quark since it has the largest coupling to the Higgs. With no additional final state particles to first order, searches for this production mode usually involve initial state radiation from the gluons or the loop. As such, the signature is at least one \gls{jet} and \ptvecmiss.


%=========================================================


\subsection{Results of previous searches}
\label{subsec:theory_hinv_prev_results}

Many previous analyses have investigated the \higgstoinv decay, in some cases from dedicated searches, but often as an afterthought or interpretation of the main analysis. \acrshort{vbf} is the most sensitive production mode. This is demonstrated in Tab.~\ref{tab:hinv_br_limits} by the upper limits attained compared to the other mechanisms. In Ref.~\citenum{Sirunyan:2018owy}, a combination was performed by \acrshort{cms} over all the productions modes detailed in the table (with the exception of \ttH). Using the recent 2016 measurements as well as data taken from Run-1 and 2015, this combined observed upper limit sits at 19\,\%, while the expected is 15\,\%. With only data from Run-1 and 2015 (the previous combination) the observed and expected upper limits of 24\,\% and 23\,\%, respectively, were found~\cite{Khachatryan:2016whc}. Similar sensitivities were achieved for \acrshort{vbf} and \ttH with the corresponding datasets by \acrshort{atlas}~\cite{ATLAS:2020kdi}.

\begin{table}[H]  % Forcing so it appears on same page and not after SVJ subsec begins
    \centering
    \begin{tabular}{lllcc}
        \toprule
        Targeted mode & Analysis & Final state & Observed (\%) & Expected (\%)\\\midrule
        \acrshort{vbf} & Ref.~\citenum{Sirunyan:2018owy} & $\text{\acrshort{vbf}-\glspl{jet}} + \ptvecmiss$ & 33 & 25 \\
        $\VH(\HepProcess{\PVec \to \Pquark\APquark})$ & Ref.~\citenum{Sirunyan:2017jix} & $\PVec(\HepProcess{\to \Pquark\APquark}) + \ptvecmiss$ & 50 & 48 \\
        $\ggH$ & Ref.~\citenum{Sirunyan:2017jix} & $\text{\glspl{jet}} + \ptvecmiss$ & 66 & 59 \\
        $\ttH$ & Ref.~\citenum{CMS-PAS-HIG-18-008} & $\ttbar(\HepProcess{\to \text{\glspl{jet}}}) + \ptvecmiss$ & 85 & 73 \\
        \bottomrule
    \end{tabular}
    \caption[The most recent searches for invisibly decaying Higgs bosons in hadronic channels with 2016 data from CMS, and the achieved upper limits on the \higgstoinv branching ratio at 95\,\% confidence level]{The most recent searches for invisibly decaying Higgs bosons in hadronic channels with 2016 data from \acrshort{cms}, and the achieved upper limits on the \higgstoinv branching ratio at 95\,\% confidence level.}
    \label{tab:hinv_br_limits}
\end{table}


%=========================================================


\section{Searches for semi-visible jets}
\label{sec:theory_svj}

Many searches for dark matter presume it is a \acrshort{wimp}-like particle because of the considerations discussed in Chpt.~\ref{sec:theory_dm_landscape}. In the \acrshort{lhc}, the signatures of \glspl{wimp} would be driven by large missing transverse momentum recoiling from visible matter in the event. Monojet~\cite{Khachatryan:2014rra} and dijet~\cite{Sirunyan:2016iap} searches are able to exploit this, for example. However, no sign of \glspl{wimp} have been observed yet. Thankfully, a boundless supply of alternative theories exist, with possible signatures equally as varied. Though the \ptvecmiss could still be one of the characteristics by which the dark matter can be inferred, a plethora of topologies and discriminating observables are possible. The dynamics that govern dark matter may be confined to a \emph{dark sector} or \emph{hidden sector}, inhabited by new forces and particles.

A dark sector may be largely inaccessible, as in some Hidden Valley scenarios,\footnote{A Hidden Valley is a schema where the \acrlong{sm} is extended by a non-abelian group. \acrshort{sm} particles are uncharged under this group. The new, light particles from this extension are the opposite: charged under the new group and neutral under the \acrshort{sm} gauge group. A heavy mediator carries both charges, acting as a portal between the \acrlong{sm} and Hidden Valley particles.} but communicate with the visible sector through a portal interaction~\cite{Strassler:2006im}. An example from \acrshort{sm} particles could be the Higgs boson bridging the visible and hidden sectors, as mentioned in Chpt.~\ref{sec:theory_higgs_to_inv}. Many interesting and novel signatures can be probed by \acrshort{lhc} experiments from models like these. Dark forces with energy scales in the tens of \GeVns and mediator masses up to several \TeVns may be accessible. If they share parallels with the \acrlong{sm}, the mechanisms for the dark matter presence and relic density can be explained as arising from a baryon-like asymmetry.

Proposed in Refs.~\citenum{Cohen:2015toa}{}and \citenum{Cohen:2017pzm}, a strongly-coupled dark sector in a Hidden Valley is imagined with interactions analogous to \acrshort{qcd}. Its internal dynamics are described by an SU(2) gauge group with dark gluons mediating the interactions, conserving a dark colour charge. The portals allowing the dark and visible sectors to communicate can be decomposed into a leptophobic \PZprime (\schannel) and bi-fundamental $\PBifund$ (\tchannel) mediating a dark weak force. In the \tchannel case, $\PBifund$ is a representation of both the visible and dark \acrshort{qcd} gauge groups. Depictions of the processes above are given in Fig.~\ref{fig:theory_svj_portals}. In the \acrshort{lhc}, protons could collide at energies high enough to access the dark sector. From either the resonant production of a \PZprime or exchange of a $\PBifund$, dark quarks \Pqdark are produced. Below a dark confinement scale \lamDark, hadronisation takes place to coalesce them into dark hadrons. Depending on the species, some of these dark hadrons are stable (i.e., a source of dark matter), while others are unstable and decay back into visible sector particles, namely \acrlong{sm} quarks. The final state is then a shower of two \glspl{jet} each interspersed with dark matter: \glspl{svj}.

\begin{figure}[htbp]
    \centering
    \begin{subfigure}[c]{0.32\textwidth}
    \centering
        \includegraphics[width=\textwidth]{figures/svj/portals_s.pdf}
        \caption{\schannel}
    \end{subfigure}
    \hspace{0.1\textwidth}
    \begin{subfigure}[c]{0.32\textwidth}
    \centering
        \includegraphics[width=\textwidth]{figures/svj/portals_t.pdf}
        \caption{\tchannel}
    \end{subfigure}
\caption[Example Feynman diagrams for the two main production modes of semi-visible jets. A \PZprime boson mediates the \schannel process while a bi-fundamental $\PBifund$ mediates the \tchannel process]{Example Feynman diagrams for the two main production modes of \glspl{svj}. A \PZprime boson mediates the \schannel process while a bi-fundamental $\PBifund$ mediates the \tchannel process. Figure from Ref.~\citenum{Cohen:2017pzm}.}
\label{fig:theory_svj_portals}
\end{figure}


%=========================================================


\subsection{Kinematics and free parameters of the model}
\label{subsec:theory_svj_free_params}

The kinematics of \glspl{svj} are heavily influenced by the following free parameters of the model: the mass of the mediator (\mZprime or $\mBifund$), the dark coupling strength (\aDark), the dark quark mass (\mqdark), and the invisible fraction (\rinv).

\begin{easylist}[itemize]
    \easylistprops
    & \mZprime/$\mBifund$: Since the energies of the colliding protons have an upper limit, the conservation of energy (or momentum) imposes one for the on-shell production/exchange of the mediator particle. In the \schannel process, production of the \PZprime is resonant. Consequently, its mass is possible to recover by calculating the dijet mass \mjj or transverse mass \mT.

    & \aDark: In Ref.~\citenum{Cohen:2017pzm}, this is defined as $\gqdark^2/ 4\pi$ (where $\gqdark$ is the coupling constant between the dark quarks and mediator). Analogous to \acrshort{qcd}, the dark coupling runs as a function of the energy scale, influencing \lamDark. At 1\TeV,
    \begin{equation}
        \lamDark = 1000 \ [\GeVns] \exp( \frac{-2\pi}{\aDark b} )
        \label{eq:lambda_dark}
    \end{equation}
    where $b = \frac{11}{3}\Nc - \frac{2}{3}\Nf$ is related to the number of dark colours and flavours, respectively.

    & \mqdark: This parameter does not directly affect much, but is related to the dark hadron mass ($\mDark = \text{2}\mqdark$) and \lamDark. The combination of the two properties affects the shower dynamics. Note that while Ref.~\citenum{Cohen:2017pzm}{}describes some of these to be insensitive, a parameter scan over these two variables are necessary in the study described in Chpt.~\ref{chap:svj}.

    & \rinv: This is defined as the fraction of produced dark particles that remain invisible, at least over timescales where they would interact with a detector. When generating simulated samples, \rinv can be interpreted as the \emph{probability} of a dark hadron being stable. While this variable is not inherent within the model, it is one that can parametrise many underlying components. As a result, visualisation of the shower and direction of \ptvecmiss is much more intuitive, as demonstrated in Figs.~\ref{fig:theory_svj_rinv} and \ref{fig:theory_svj_met_dir}, respectively. A large value of \rinv would yield a similar final state to a \acrshort{wimp} search.
\end{easylist}

\begin{figure}[htbp]
    \centering
    \includegraphics[width=0.75\textwidth]{figures/svj/r_inv.pdf}
    \caption[The constituents of a semi-visible jet as a function of its invisible fraction]{The constituents of a \gls{svj} as a function of its invisible fraction \rinv. The green dashed lines signify visibly decaying dark hadrons, blue the \acrshort{sm} quarks, and pink the stable dark hadrons. Figure taken from Ref.~\citenum{Cohen:2017pzm}.}
    \label{fig:theory_svj_rinv}
\end{figure}

\begin{figure}[htbp]
    \centering
    \includegraphics[width=0.85\textwidth]{figures/svj/metfigure.pdf}
    \caption[The typical direction of the missing transverse energy relative to the semi-visible jets as a function of the invisible fraction \rinv]{The typical direction of the missing transverse energy \ETslash\xspace (or \ptvecmiss) relative to the \glspl{svj} as a function of their invisible fraction \rinv. The green dashed lines signify visibly decaying dark hadrons, blue the \acrshort{sm} quarks, and pink the stable dark hadrons. Figure from Ref.~\citenum{Cohen:2017pzm}.}
    \label{fig:theory_svj_met_dir}
\end{figure}

In the studies of simulation for \glspl{svj} in Chpt.~\ref{chap:svj}, only the \schannel process has been analysed with \acrshort{lhc} data with publication on the horizon. Generator studies have been additionally performed for the \tchannel interaction and the analysis is underway. In the \schannel search, mediator masses of up to several \TeVns are accessible, and intermediate values of \rinv are most sensitive. Hence, the typical signature is a dijet pair with each \gls{jet} likely to contain a different invisible fraction, leading to the \ptvecmiss aligned with one of the \glspl{jet}. \glspl{wimp}, on the other hand, completely recoil from the visible matter, and so \glspl{jet} may be more collimated with small separation. The \ptvecmiss is also larger in magnitude and likely more isolated. The phase space exploited by this model is often rejected by dark matter searches since the final state can be easily mimicked by mismeasured \acrshort{qcd}. A sizeable background from this process would therefore be present. However, \gls{jet} substructure techniques and machine learning algorithms have developed rapidly in the recent years, and it is possible to disentangle signal and background with some certainty~\cite{GiorgiaRaucoThesis}.

One interesting aspect of the model is the potential for signatures with displaced vertices, so called long-lived particles or \emph{emerging jets} on account of the decay to visible states occurs a sufficient distance from the primary vertex. Some searches have already been performed for this final state from a different interpretation of a strongly-coupled dark force~\cite{Sirunyan:2018njd} to \acrlong{susy} contexts~\cite{SUS16038published}. These are not considered in Chpt.~\ref{chap:svj}, so the dark hadrons are assumed to decay promptly. Long-lived interpretations have been noted as possible extensions to the search, however.
