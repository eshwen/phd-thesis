%
% File: abstract.tex
% Author: Victor Brena-Medina
% Description: Contains the text for thesis abstract
%
% UoB guidelines:
%
% Each copy must include an abstract or summary of the dissertation in not more than 300 words, on one side of A4, which should be single-spaced in a font size in the range 10 to 12. If the dissertation is in a language other than English, an abstract in that language and an abstract in English must be included.

\chapter*{Abstract}
\begin{SingleSpace}
\initial{D}ark matter is a poorly-understood phenomenon in nature. Though substantial evidence corroborates its existence, only few characteristics have been determined. One goal of the Large Hadron Collider (LHC) at CERN is to produce dark matter in high energy proton-proton collisions, potentially allowing insight into its currently-mysterious origins. Many models have been postulated regarding its nature, two of which are investigated in this thesis: invisible decays of the Higgs boson, and the production of semi-visible jets. The data used is from the LHC Run-2 era and recorded by the CMS experiment, corresponding to an integrated luminosity of 137\fbinv at a centre of mass energy of 13\TeV.

The branching ratio of the Higgs boson to invisible states is predicted to be 0.1\,\% in the standard model of particle physics. Enhancements from a coupling to dark matter may be observable at the LHC. A search is performed in final states comprising jets and missing transverse momentum targeting the \ttH and \VH Higgs boson production modes. With the full Run-2 dataset from CMS, no significant deviation from the standard model is observed. Results are presented as an upper limit on the measured cross section times branching ratio over the standard model Higgs boson cross section at the 95\,\% confidence level. For the \ttH-tagged events, observed and expected limits of 0.56 and 0.50, respectively, are achieved. For the \VH-tagged events, observed and expected limits of 0.32 and 0.22, respectively, are found. A combined Run-2 limit of 0.28 is observed compared to 0.20 expected. These results are interpreted in simplified dark matter scenarios.

Dark matter may exist in a Hidden Valley dark sector connected to the visible universe via a leptophobic mediator. Analogous to QCD, dark quarks may be produced at the LHC, hadronising and decaying into a mixture of visible and invisible particles: a semi-visible jet. The behaviour of simulated signal from $s$- and \tchannel production modes in the CMS detector---and variables discriminating it from background---is explored. The transverse mass of the dijet system was found to be the most effective.
\end{SingleSpace}
\clearpage
