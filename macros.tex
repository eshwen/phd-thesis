\newcommand{\keywords}[1]{\par\noindent{\small{\bf Keywords:} #1}} % Defines keywords small section

% An initial of the very first character of the content
\usepackage{lettrine}
\newcommand{\initial}[1]{%
	\lettrine[lines=3,lhang=0.33,nindent=0em]{
		\color{gray}
            {\textsc{#1}}}{}}

% Generic
\renewcommand{\_}{\texttt{\char`_}}
\newcommand{\plusjets}{\ensuremath{+ \text{jets}}\xspace}
\newcommand{\jone}{\ensuremath{\mathrm{j}_1}\xspace}
\newcommand{\jtwo}{\ensuremath{\mathrm{j}_2}\xspace}
\newcommand{\PVec}{\ensuremath{\mkern-2mu}\HepParticle{V}{}{}\xspace} % couldn't find command for generic V boson so created one. Removing a small amount of leading horizontal space, otherwise with the particle in italics, enclosing in brackets or spaces before/after it look a little asymmetrical

% Electron volt units
% Re-defining ptdr symbols for better spacing, consistency and simplicity with definitions, expanding list of symbols, using math mode ^2 for consistency with other symbols
\newcommand{\eV}{\text{e\kern-0.15ex V}\xspace}
\newcommand{\eVnxs}{\text{e\kern-0.15ex V}}  % no trailing xspace (e.g., for Gev/c or GeV/c^2)
\renewcommand{\keV}{\ensuremath{\,\text{k\eV}}\xspace}
\newcommand{\keVns}{\ensuremath{\text{k\eV}}\xspace}  % no leading thinspace
\renewcommand{\MeV}{\ensuremath{\,\text{M\eV}}\xspace}
\renewcommand{\MeVns}{\ensuremath{\text{M\eV}}\xspace}  % no leading thinspace
\renewcommand{\GeV}{\ensuremath{\,\text{G\eV}}\xspace}
\renewcommand{\GeVns}{\ensuremath{\text{G\eV}}\xspace}  % no leading thinspace
\renewcommand{\gev}{\GeV}
\newcommand{\Tkern}{\text{T\kern-0.2ex}}  % reducing space between the T and eV in TeV
\renewcommand{\TeV}{\ensuremath{\,\text{\Tkern\eV}}\xspace}
\renewcommand{\TeVns}{\ensuremath{\text{\Tkern\eV}}\xspace}  % no leading thinspace
\renewcommand{\PeV}{\ensuremath{\,\text{P\eV}}\xspace}
\newcommand{\PeVns}{\ensuremath{\text{P\eV}}\xspace}  % no leading thinspace

\newcommand{\overc}{\ensuremath{\text{\hspace{-0.16em}/\hspace{-0.08em}}c}}
\newcommand{\overcc}{\ensuremath{\overc^2}}
\newcommand{\eVc}{\ensuremath{\,\text{\eVnxs\overc}}\xspace}
\newcommand{\eVcc}{\ensuremath{\,\text{\eVnxs\overcc}}\xspace}
\renewcommand{\keVc}{\ensuremath{\,\text{k\eVnxs\overc}}\xspace}
\renewcommand{\MeVc}{\ensuremath{\,\text{M\eVnxs\overc}}\xspace}
\renewcommand{\GeVc}{\ensuremath{\,\text{G\eVnxs\overc}}\xspace}
\renewcommand{\GeVcns}{\text{G\eVnxs\overc}\xspace}  % no leading thinspace
\renewcommand{\TeVc}{\ensuremath{\,\text{\Tkern\eVnxs\overc}}\xspace}
\renewcommand{\keVcc}{\ensuremath{\,\text{k\eVnxs\overcc}}\xspace}
\renewcommand{\MeVcc}{\ensuremath{\,\text{M\eVnxs\overcc}}\xspace}
\renewcommand{\GeVcc}{\ensuremath{\,\text{G\eVnxs\overcc}}\xspace}
\renewcommand{\GeVccns}{\text{G\eVnxs\overcc}\xspace}  % no leading thinspace
\renewcommand{\TeVcc}{\ensuremath{\,\text{\Tkern\eVnxs\overcc}}\xspace}

% Rationale for when to use \text{} and when to use \mathrm{} in physics symbols:
%   - In math mode, wrapping something \text{} will render it in the main document font (i.e., garamond from the 'garamondx' package).
%   - In math mode, wrapping something in \mathrm{} will render it upright in the math font (i.e., the font from the 'amsfonts' package).
%   - \text{} should be used for things like control regions where the text from the macro is supposed to flow into the main text. Then it looks natural and consistent.
%   - \mathrm{} should be used for subscripts and superscripts in symbols so it looks natural and consistent in that context. Even for sub/superscripts longer than one letter, e.g., the "min" in biased delta phi, it looks more natural if the font matches the rest of the symbol and equation.

% Particles are typeset using the 'hepnames' package. If given the option 'italic', it will render the particles in italics. So if I want to change style, only need to edit the line that imports the package, not the macros.

% Higgs to inv. modes. Using negative horizontal space so that, with the symbols rendered in italics, the spacing looks better
\renewcommand{\ttbar}{\ensuremath{\Ptop\APtop}\xspace}
\newcommand{\ttH}{\ensuremath{\ttbar\mkern-1mu\PH}\xspace}
\newcommand{\ggF}{\ensuremath{\Pgluon\mkern-1mu\Pgluon\mathrm{F}}\xspace}
\newcommand{\ggH}{\ensuremath{\Pgluon\mkern-1mu\Pgluon\mkern-1mu\PH}\xspace}
\newcommand{\VH}{\ensuremath{\PVec\mkern-2mu\PH}\xspace}
\newcommand{\WH}{\ensuremath{\PW\mkern-2mu\PH}\xspace}
\newcommand{\WplusH}{\ensuremath{\PWp\mkern-2mu\PH}\xspace}
\newcommand{\WminusH}{\ensuremath{\PWm\mkern-2mu\PH}\xspace}
\newcommand{\ZH}{\ensuremath{\PZ\mkern-2mu\PH}\xspace}

% Higgs to inv. control regions
\newcommand{\singleMuCr}{\ensuremath{\Pmu \plusjets}\xspace}
\newcommand{\doubleMuCr}{\ensuremath{\Pmu\Pmu \plusjets}\xspace}
\newcommand{\singleEleCr}{\ensuremath{\Pe \plusjets}\xspace}
\newcommand{\doubleEleCr}{\ensuremath{\Pe\Pe \plusjets}\xspace}
\newcommand{\singlePhotonCr}{\ensuremath{\Pphoton \plusjets}\xspace}
\newcommand{\tightMuon}{\ensuremath{\Pmu_{\mathrm{tight}}}\xspace}
\newcommand{\looseMuon}{\ensuremath{\Pmu_{\mathrm{loose}}}\xspace}
\newcommand{\tightEle}{\ensuremath{\Pe_{\mathrm{tight}}}\xspace}
\newcommand{\vetoEle}{\ensuremath{\Pe_{\mathrm{veto}}}\xspace}
\newcommand{\mediumPhoton}{\ensuremath{\Pphoton_{\mathrm{medium}}}\xspace}
\newcommand{\loosePhoton}{\ensuremath{\Pphoton_{\mathrm{loose}}}\xspace}
\newcommand{\vlooseTau}{\ensuremath{\Ptau_{\mathrm{v. \ loose}}}\xspace}

% Variables/physics symbols
\newcommand{\doubleMuMass}{\ensuremath{m_{\Pmu\Pmu}}\xspace}
\newcommand{\doubleEleMass}{\ensuremath{m_{\Pe\Pe}}\xspace}
\newcommand{\doubleLepMass}{\ensuremath{m_{\Plepton\Plepton}}\xspace}

\newcommand{\etmiss}{\MET}
\newcommand{\met}{\MET}
\newcommand{\htmiss}{\mht}
\newcommand{\alphat}{\ensuremath{\alpha_{\mathrm{T}}}\xspace}
\newcommand{\alphaT}{\alphat}
\newcommand{\biasedDPhi}{\ensuremath{\Delta\phi^*_{\mathrm{min}}}\xspace}
\newcommand{\pT}{\pt}
\newcommand{\mt}{\mT}

\newcommand{\mTsup}[1]{\ensuremath{\mT^{#1}}\xspace}  % to add a superscript in mT. Call like \mTsup{foo} to get m_T^{foo}
\newcommand{\mtMuon}{\ensuremath{\mTsup{\Pmu}}\xspace}
\newcommand{\mtElectron}{\ensuremath{\mTsup{\Pe}}\xspace}
\newcommand{\mjj}{\ensuremath{m_{\mathrm{jj}}}\xspace}
\newcommand{\ptsup}[1]{\ensuremath{\pt^{#1}}\xspace}  % to add a superscript in pT
\newcommand{\etasub}[1]{\ensuremath{\eta_{#1}}\xspace} % to add a subscript in eta
\newcommand{\abseta}{\ensuremath{\abs{\eta}}\xspace}
\newcommand{\absetasub}[1]{\ensuremath{\abs{\eta_{#1}}}\xspace}
\newcommand{\nsub}[1]{\ensuremath{n_{#1}}\xspace}  % to add a subscript in n
\newcommand{\ptjone}{\ensuremath{\ptsup{\jone}}\xspace}
\newcommand{\ptjtwo}{\ensuremath{\ptsup{\jtwo}}\xspace}
\newcommand{\etajone}{\ensuremath{\etasub{\jone}}\xspace}
\newcommand{\etajtwo}{\ensuremath{\etasub{\jtwo}}\xspace}
\newcommand{\njet}{\ensuremath{\nsub{\mathrm{jet}}}\xspace}
\newcommand{\nbjet}{\ensuremath{\nsub{\Pbottom}}\xspace}
\newcommand{\nBoostedTop}{\ensuremath{\nsub{\Ptop}}\xspace}
\newcommand{\nBoostedV}{\ensuremath{\nsub{\PVec}}\xspace}

% Angular variables
\newcommand{\omegaHat}{\ensuremath{\hat{\omega}_{\text{min}}}\xspace}
\newcommand{\omegaTilde}{\ensuremath{\tilde{\omega}_{\text{min}}}\xspace}
\newcommand{\minChi}{\ensuremath{\chi_{\text{min}}}\xspace}
\newcommand{\mindphi}{\ensuremath{\Delta\phi_{\text{min}}}\xspace}
\newcommand{\mindphiAB}[2]{\ensuremath{\Delta\phi_{\text{min}}(#1, \ #2)}\xspace}
\newcommand{\mindphiJetMet}{\ensuremath{\mindphiAB{\text{j}}{\ptmiss}}\xspace}
\newcommand{\dphiTj}{\ensuremath{\mindphiAB{\text{j}_{1, 2}}{\ptmiss}}\xspace}
\newcommand{\dphiFj}{\ensuremath{\mindphiAB{\text{j}_{1, 2, 3, 4}}{\ptmiss}}\xspace}

% Semi-visible jets variables. Taken from AN-19-061
%\newcommand{\metmt}{\ensuremath{\MET/\mt}\xspace}
\newcommand{\metmt}{\ensuremath{R_{\mathrm{T}}}\xspace}
\renewcommand{\PZprime}{\ensuremath{{\PZ}^{\prime}}\xspace}
\newcommand{\PZprimesup}[1]{\ensuremath{{\PZ}^{\prime#1}}\xspace}
\newcommand{\mZprime}{\ensuremath{m_{\PZprime}}\xspace}
\newcommand{\sigmaZprime}{\ensuremath{\sigma_{\PZprime}}\xspace}
\newcommand{\mDark}{\ensuremath{m_{\mathrm{dark}}}\xspace}
\newcommand{\aDark}{\ensuremath{\alpha_{\mathrm{dark}}}\xspace}
\newcommand{\lamDark}{\ensuremath{\Lambda_{\mathrm{dark}}}\xspace}
\newcommand{\aDarkPeak}{\ensuremath{\aDark^{\text{peak}}}\xspace}
\newcommand{\lamDarkPeak}{\ensuremath{\lamDark^{\text{peak}}}\xspace}
\newcommand{\aDarkHigh}{\ensuremath{\aDark^{\text{high}}}\xspace}
\newcommand{\aDarkLow}{\ensuremath{\aDark^{\text{low}}}\xspace}
\newcommand{\rinv}{\ensuremath{r_{\mathrm{inv}}}\xspace}
\newcommand{\Pqdark}{\ensuremath{\chi}\xspace}
\newcommand{\mqdark}{\ensuremath{m_{\Pqdark}}\xspace}
\newcommand{\Paqdark}{\ensuremath{\overline{\chi}}\xspace}
\newcommand{\PqdarkO}{\ensuremath{\chi_{1}}\xspace}
\newcommand{\PqdarkT}{\ensuremath{\chi_{2}}\xspace}
\newcommand{\Pgdark}{\ensuremath{\cPg_{\text{dark}}}\xspace}
\newcommand{\Ppidark}{\ensuremath{\pi_{\text{dark}}}\xspace}
\newcommand{\PpidarkDM}{\ensuremath{\Ppidark^{\text{DM}}}\xspace}
\newcommand{\Prhodark}{\ensuremath{\rho_{\text{dark}}}\xspace}
\newcommand{\PrhodarkDM}{\ensuremath{\Prhodark^{\text{DM}}}\xspace}
\newcommand{\gq}{\ensuremath{g_{\Pquark}}\xspace}  % AN uses additional ^{\PZprime} in symbol
\newcommand{\gqdark}{\ensuremath{g_{\Pqdark}}\xspace}  % AN uses additional ^{\PZprime} in symbol
\newcommand{\Nc}{\ensuremath{N_{c}}\xspace}
\newcommand{\Nf}{\ensuremath{N_{f}}\xspace}
\newcommand{\mb}{\ensuremath{m_{\cPqb}}\xspace}
\newcommand{\mc}{\ensuremath{m_{\cPqc}}\xspace}
\newcommand{\mq}{\ensuremath{m_{\cPq}}\xspace}
\newcommand{\Nstable}{\ensuremath{N_{\text{stable}}}\xspace}
\newcommand{\Nunstable}{\ensuremath{N_{\text{unstable}}}\xspace}
\newcommand{\dijetDeta}{\ensuremath{\Delta\eta(j_{1},j_{2})}\xspace}
\newcommand{\dijetMindphi}{\ensuremath{\Delta\phi_{\mathrm{min}}(j_{1,2}, \met)}\xspace}
\newcommand{\schannel}{\ensuremath{s\text{-channel}}\xspace}
\newcommand{\tchannel}{\ensuremath{t\text{-channel}}\xspace}
\newcommand{\PBifund}{\ensuremath{\Phi}\xspace}
\newcommand{\mBifund}{\ensuremath{m_{\PBifund}}\xspace}

% Background processes
\newcommand{\lostlepton}{\ensuremath{\Plepton_{\text{lost}}}\xspace}
\newcommand{\ztonunu}{\ensuremath{\HepProcess{\PZ \to \Pnu\APnu}}\xspace}
\newcommand{\ztomumu}{\ensuremath{\HepProcess{\PZ \to \Pmu\Pmu}}\xspace}
\newcommand{\ztonunupjets}{\ensuremath{\PZ(\to \Pnu\APnu) \plusjets}\xspace}
\newcommand{\ztollpjets}{\ensuremath{\PZ(\to \Plepton\Plepton) \plusjets}\xspace}
\newcommand{\ztolplmpjets}{\ensuremath{\PZ(\to \Pleptonplus\Pleptonminus) \plusjets}\xspace}
\newcommand{\wtolnupjets}{\ensuremath{\PW(\to \Plepton\Pnu) \plusjets}\xspace}
\newcommand{\ttbarpjets}{\ensuremath{\ttbar \plusjets}\xspace}

% Software
\newcommand{\madgraph}{\MADGRAPH}
\newcommand{\MADGRAPHFULL}{\MGvATNLO~2.6.0\xspace}  % What I used for SVJ
\newcommand{\MADSPIN}{\textsc{MadSpin}\xspace}
\newcommand{\PYTHIAEIGHT}{\textsc{pythia8}\xspace}
\newcommand{\FEYNRULES}{\textsc{FeynRules}\xspace}
\newcommand{\madanalysis}{\textsc{MadAnalysis}\xspace}
\newcommand{\rivet}{\textsc{Rivet}\xspace}
\newcommand{\ROOT}{\textsc{root}\xspace}
\newcommand{\nanoAODtools}{\textsf{nanoAOD-tools}\xspace}
\newcommand{\deepakeight}{\textsc{DeepAK8}\xspace}
\newcommand{\deepcsv}{\textsc{DeepCSV}\xspace}
\newcommand{\fastcarpenter}{\textsc{fast-carpenter}\xspace}
\newcommand{\fastcurator}{\textsc{fast-curator}\xspace}
\newcommand{\fastplotter}{\textsc{fast-plotter}\xspace}

% Theory
\newcommand{\Dslash}{\ensuremath{\not\!\!D}\xspace}
\newcommand{\dslash}{\ensuremath{\not\!\partial}\xspace}
\newcommand{\BR}{\ensuremath{\mathcal{B}}\xspace}
\newcommand{\BRof}[1]{\ensuremath{\BR(#1)}\xspace} % branching ratio of a process

% Other (physics-related)
\newcommand{\intlumi}{\ensuremath{\lumi_{\mathrm{int.}}}\xspace}
\newcommand{\bkgsystuncert}{\ensuremath{\sigma_{B,\,\mathrm{syst.}}}\xspace}
\newcommand{\transfac}{\ensuremath{\mathcal{T}}\xspace}
\newcommand{\TF}{\transfac}
\newcommand{\LSP}{\PSneutralinoOne}
\newcommand{\comruntwo}{\ensuremath{\sqrt{s} = \text{13}\TeV}\xspace}
\newcommand{\pp}{\ensuremath{\Pp\mkern-1mu\Pp}\xspace}  % define macro with tighter spacing between each p. Looks nicer when rendering symbols in italics
\newcommand{\higgstoinv}{\ensuremath{\HepProcess{\PH \to \text{inv.}}}\xspace}
\newcommand{\xsecSI}{\ensuremath{\sigma_{\mathrm{SI}}}\xspace}
\newcommand{\ptLOne}{\ensuremath{\ptsup{{\mathrm{L1}}}}\xspace}
\newcommand{\ptRef}{\ensuremath{\ptsup{{\mathrm{ref.}}}}\xspace}
