%\title{University of Bristol Thesis Template}

% In TeXShop, compile pdflatex, bibliography and glossary in one command by selecting "pdflatexmk" from the drop-down menu next to "Typeset".
% If compilation with pdflatexmk fails, hit the "Trash Aux Files" button and recompile.

% ----------------------------------------
\RequirePackage[l2tabu]{nag}		% Warns for incorrect (obsolete) LaTeX usage
% ----------------------------------------

% ----------------------------------------
% File: memoirthesis.tex
% Author: Victor Brena
% Description: Contains the thesis template using memoir class, which is mainly based on book class but permits better control of  chapter styles for example. This template is an adaptation and modification of Oscar's.
% 
% Memoir is a flexible class for typesetting poetry, fiction, non-fiction and mathematical works as books, reports, articles or manuscripts. CTAN repository is found at: http://www.ctan.org/tex-archive/macros/latex/contrib/memoir/
%
%
% UoB guidelines for thesis presentation were found at: http://www.bris.ac.uk/esu/pg/pgrcop11-12topic.pdf#page=49
%
% UoB guidelines:
%
% The dissertation must be printed on A4 white paper. Paper up to A3 may be used for maps, plans, diagrams and illustrative material. Pages (apart from the preliminary pages) should normally be double-sided.
%
% Memoir class loads useful packages by default (see manual).
\documentclass[a4paper,12pt,leqno,openbib,oldfontcommands]{memoir} % add 'draft' to turn draft option on (see below)
% ----------------------------------------

% ----------------------------------------
% Adding metadata:
\usepackage{datetime}
\usepackage{ifpdf}
\ifpdf
\pdfinfo{
   /Author Eshwen Bhal
   /Title (PhD Thesis)
   /Keywords (One; Two;Three)
   /CreationDate (D:\pdfdate)
}
\fi
% When draft option is on. 
\ifdraftdoc 
	\usepackage{draftwatermark} % Sets watermarks up.
	\SetWatermarkScale{0.3}
	\SetWatermarkText{\bf Draft: \today}
\fi
% ----------------------------------------

% ----------------------------------------
% Declare figure/table as a subfloat.
\newsubfloat{figure}
\newsubfloat{table}
% ----------------------------------------

% ----------------------------------------
% Better page layout for A4 paper, see memoir manual.
\settrimmedsize{297mm}{210mm}{*}
\setlength{\trimtop}{0pt} 
\setlength{\trimedge}{\stockwidth} 
\addtolength{\trimedge}{-\paperwidth} 
\settypeblocksize{634pt}{448.13pt}{*} 
\setulmargins{4cm}{*}{*} 
\setlrmargins{*}{*}{1.5} 
\setmarginnotes{17pt}{51pt}{\onelineskip} 
\setheadfoot{\onelineskip}{2\onelineskip} 
\setheaderspaces{*}{2\onelineskip}{*} 
\checkandfixthelayout
% ----------------------------------------

% ----------------------------------------
\frenchspacing % single space after a sentence
% ----------------------------------------

% ----------------------------------------
% Font with math support: New Century Schoolbook
%\usepackage{fouriernc} % default font for template
%\usepackage{courier} % this is the font I use in my lab book
\usepackage[T1]{fontenc} % To encode the font (glyphs/characters as bits)
\usepackage[utf8]{inputenc} % Needed to encode non-english characters directly for mac
\usepackage{garamondx} % this is the font of the main text
% ----------------------------------------

% ----------------------------------------
% UoB guidelines:
%
% Text should be in double or 1.5 line spacing, and font size should be chosen to ensure clarity and legibility for the main text and for any quotations and footnotes. Margins should allow for eventual hard binding.
%
% Note: This is automatically set by memoir class. Nevertheless \OnehalfSpacing enables 1.5 line spacing but leaves single spaced for captions for instance. Double spacing can be achieved with \linespread{1.667} (value isn't 2 because of https://tex.stackexchange.com/questions/30073/why-is-the-linespread-factor-as-it-is)
\OnehalfSpacing
%
% Sets numbering division level
\setsecnumdepth{subsection} 
\maxsecnumdepth{subsubsection}
%
% Chapter style (taken and slightly modified from Lars Madsen Memoir Chapter 
% Styles document
\usepackage{calc,soul}
\makeatletter 
\newlength\dlf@normtxtw 
\setlength\dlf@normtxtw{\textwidth} 
\newsavebox{\feline@chapter} 
\newcommand\feline@chapter@marker[1][4cm]{%
	\sbox\feline@chapter{% 
		\resizebox{!}{#1}{\fboxsep=1pt%
			\colorbox{gray}{\color{white}\thechapter}% 
		}}%
		\rotatebox{90}{% 
			\resizebox{%
				\heightof{\usebox{\feline@chapter}}+\depthof{\usebox{\feline@chapter}}}% 
			{!}{\scshape\so\@chapapp}}\quad%
		\raisebox{\depthof{\usebox{\feline@chapter}}}{\usebox{\feline@chapter}}%
} 
\newcommand\feline@chm[1][4cm]{%
	\sbox\feline@chapter{\feline@chapter@marker[#1]}% 
	\makebox[0pt][c]{% aka \rlap
		\makebox[1cm][r]{\usebox\feline@chapter}%
	}}
\makechapterstyle{daleifmodif}{
	\renewcommand\chapnamefont{\normalfont\Large\scshape\raggedleft\so} 
	\renewcommand\chaptitlefont{\normalfont\Large\bfseries\scshape} 
	\renewcommand\chapternamenum{} \renewcommand\printchaptername{} 
	\renewcommand\printchapternum{\null\hfill\feline@chm[2.5cm]\par} 
	\renewcommand\afterchapternum{\par\vskip\midchapskip} 
	\renewcommand\printchaptertitle[1]{\color{gray}\chaptitlefont\raggedleft ##1\par}
} 
\makeatother 
\chapterstyle{daleifmodif}
% Ensure that using bold in headings applies to math symbols as well
\makeatletter
\g@addto@macro\bfseries\boldmath
\makeatother
% ----------------------------------------

% ----------------------------------------
% UoB guidelines:
%
% The pages should be numbered consecutively at the bottom centre of the page.
\makepagestyle{myvf} 
\makeoddfoot{myvf}{}{\thepage}{} 
\makeevenfoot{myvf}{}{\thepage}{} 
\makeheadrule{myvf}{\textwidth}{\normalrulethickness} 
\makeevenhead{myvf}{\small\textsc{\leftmark}}{}{} 
\makeoddhead{myvf}{}{}{\small\textsc{\rightmark}}
\pagestyle{myvf}
%
% Oscar's command (it works): Fills blank pages until next odd-numbered page. Used to emulate single-sided frontmatter. This will work for title, abstract and declaration. Though the contents sections will each start on an odd-numbered page they will spill over onto the even-numbered pages if extending beyond one page (hopefully, this is ok).
\newcommand{\clearemptydoublepage}{\newpage{\thispagestyle{empty}\cleardoublepage}}
% ----------------------------------------

% ----------------------------------------
% The import command enables each chapter tex file to use relative paths when accessing supplementary files. For example, to include chapters/brewing/images/figure1.png from chapters/brewing/brewing.tex we can use \includegraphics{images/figure1} instead of \includegraphics{chapters/brewing/images/figure1}
\usepackage{import}
% ----------------------------------------

% ----------------------------------------
% Add other packages needed for chapters here. For example:
\usepackage{ptdr-definitions} 				% CMS definitions and styles. Needs to come first so definitions from other packages don't cause conflicts
\usepackage{lipsum}						% Needed to create dummy text
\usepackage{mathtools} 					% Loads the "amsmath" package, and contains other mathematical symbols and environments
\usepackage{amssymb}					% Loads the "amsfonts" package for proper rendering of math, and contains additional math symbols
\usepackage{mathrsfs}					% Even more math symbols
\usepackage{layouts}					% Layout diagrams
\usepackage{graphicx}					% Calls figure environment, i.e., starting figures with \includegraphics
\usepackage{longtable,rotating}				% Long tab environments including rotation
\usepackage{colortbl}					% Makes coloured tables
\usepackage{float}						% Helps to place figures, tables, etc. 
\usepackage{verbatim}					% Permits pre-formated text insertion
\usepackage{latexsym}					% Extra symbols
\usepackage[square,numbers,
		     sort&compress]{natbib}		% Calls bibliography commands 
\usepackage{color}                    			% Creates coloured text and background
\usepackage[spanish,english]{babel}			% For languages characters and hyphenation
\usepackage{enumerate}					% For enumeration counter
\usepackage{footnote}					% For footnotes
\usepackage{microtype}					% Makes pdf look better (affects font and line spacing. fewer overfull/underfull hbox warnings)
\usepackage{rotfloat}					% For rotating and float environments as tables, figures, etc. 
\usepackage{alltt}						% LaTeX commands are not disabled in verbatim-like environment
\usepackage[version=0.96]{pgf}			% PGF/TikZ is a tandem of languages for producing vector graphics from a 
\usepackage{tikz}						% geometric/algebraic description.
\usetikzlibrary{arrows,shapes,snakes,
		       automata,backgrounds,
		       petri,topaths}				% To use diverse features from tikz

% Esh's extra packages from here
\usepackage{physics} 					% for physics symbols, such as differential symbols
\usepackage{float} 						% to fix figures in place
\usepackage{cprotect}					% so I can use \verb and other commands within captions
\usepackage{bbold} 						% for mathematical symbols like the characters for special sets
\usepackage[ampersand]{easylist} 			% for bullet points and lists, etc. New entry starts with & character
\usepackage{cancel} 					% for strike-throughs for missing energy/momentum, etc.
\usepackage{maybemath} 				% required for hepparticles
\usepackage{hepparticles} 				% required for hepnicenames
\usepackage{hepnicenames} 				% standard symbols for HEP particles
\usepackage{spverbatim} 					% for linebreaks in the verbatim environment
\usepackage{tabularx} 					% for multii-line cells in tables
\usepackage{multirow} 					% for nested rows in tables
\usepackage{siunitx} 					% for aligning numbers properly in tables (column type is "S")
\usepackage[normalem]{ulem} 				% for underlining over multiple lines. Need "normalem" option otherwise \emph underlines
\usepackage{pdflscape}					% for landscape-oriented tables
\usepackage{xfrac} 						% for nice-looking diagonal (a/b) fractions
\usepackage{booktabs}					% for table formatting macros. other good formatting tips in the package manual

\usepackage{array} 						% for more flexibility with tables
\newcolumntype{M}[1]{>{\centering\arraybackslash}m{#1}} % for text in table cells to be centred

\usepackage{hyperref} 					% to hyperlink references, contents, etc. Change colours if required
\hypersetup{
    colorlinks=true,
    linkcolor=black,
    filecolor=Fuchsia,
    urlcolor=red,
    citecolor=cyan,
}
\usepackage{memhfixc}					% Must be used on memoir document class after hyperref
% ----------------------------------------

% ----------------------------------------
% Create the glossary for terms and acronyms
\usepackage[acronym, toc, nonumberlist]{glossaries}
\makeglossaries
\newglossaryentry{latex}
{
        name=latex,
        description={Is a mark up language specially suited for 
scientific documents}
}

\newacronym{lhc}{LHC}{Large Hadron Collider}
% Create indexes for Table of Contents, List of Figures, List of Tables and Index
\makeindex
% ----------------------------------------

% ----------------------------------------							
% Reduce widows (the last line of a paragraph at the start of a page) and orphans (the first line of paragraph at the end of a page)
\widowpenalty=1000
\clubpenalty=1000
% ----------------------------------------

% ----------------------------------------
% New command definitions for my thesis
%
\newcommand{\keywords}[1]{\par\noindent{\small{\bf Keywords:} #1}} % Defines keywords small section
\newcommand{\pgftextcircled}[1]{                                                                    % Defines encircled text
    \setbox0=\hbox{#1}%
    \dimen0\wd0%
    \divide\dimen0 by 2%
    \begin{tikzpicture}[baseline=(a.base)]%
        \useasboundingbox (-\the\dimen0,0pt) rectangle (\the\dimen0,1pt);
        \node[circle,draw,outer sep=0pt,inner sep=0.1ex] (a) {#1};
    \end{tikzpicture}
}

\renewcommand{\_}{\texttt{\char`_}}
\newcommand{\madgraph}{\MADGRAPH}
\newcommand{\madanalysis}{\textsc{MadAnalysis}\xspace}
\newcommand{\rivet}{\textsc{Rivet}}
\newcommand{\etmiss}{\MET}
\newcommand{\met}{\MET}
\newcommand{\htmiss}{\mht}
\newcommand{\alphat}{\ensuremath{\alpha_{\mathrm{T}}}\xspace}
\newcommand{\alphaT}{\alphat}
\newcommand{\biasedDPhi}{\ensuremath{\Delta\phi^*_{\mathrm{min}}}\xspace}
\newcommand{\pT}{\pt}
\newcommand{\eV}{\text{e\kern-0.15ex V}\xspace}
\newcommand{\LSP}{\ensuremath{\tilde{\chi}_1^0}\xspace}
\newcommand{\rinv}{\ensuremath{r_{\mathrm{inv}}}\xspace}
\newcommand{\comruntwo}{\ensuremath{\sqrt{s} = 13\TeV}\xspace}
\newcommand{\higgstoinv}{\ensuremath{H \rightarrow \text{inv.}}\xspace}

% My caption style
\newcommand{\mycaption}[2][\@empty]{
	\captionnamefont{\scshape} 
	\changecaptionwidth
	\captionwidth{0.9\linewidth}
	\captiondelim{.\:} 
	\indentcaption{0.75cm}
	\captionstyle[\centering]{}
	\setlength{\belowcaptionskip}{10pt}
	\ifx \@empty#1 \caption{#2}\else \caption[#1]{#2}
}

% My subcaption style
\newcommand{\mysubcaption}[2][\@empty]{
	\subcaptionsize{\small}
	\hangsubcaption
	\subcaptionlabelfont{\rmfamily}
	\sidecapstyle{\raggedright}
	\setlength{\belowcaptionskip}{10pt}
	\ifx \@empty#1 \subcaption{#2}\else \subcaption[#1]{#2}
}

% An initial of the very first character of the content
\usepackage{lettrine}
\newcommand{\initial}[1]{%
	\lettrine[lines=3,lhang=0.33,nindent=0em]{
		\color{gray}
     		{\textsc{#1}}}{}}
% ----------------------------------------

% ----------------------------------------
% Hyphenation for some words
\hyphenation{res-pec-tively}
\hyphenation{mono-ti-ca-lly}
\hyphenation{hypo-the-sis}
\hyphenation{para-me-ters}
\hyphenation{sol-va-bi-li-ty}
% ----------------------------------------
\begin{document}
% UoB guidlines:
%
% Preliminary pages
% 
% The five preliminary pages must be the Title Page, Abstract, Dedication and Acknowledgements, Author's Declaration and Table of Contents. These should be single-sided.
% 
% Table of contents, list of tables and illustrative material
% 
% The table of contents must list, with page numbers, all chapters, sections and subsections, the list of references, bibliography, list of abbreviations and appendices. The list of tables and illustrations should follow the table of contents, listing with page numbers the tables, photographs, diagrams, etc., in the order in which they appear in the text.


\frontmatter
\pagenumbering{roman}

%
% File: title.tex
% Author: Eshwen Bhal
% Description: Contains the title page
%
% UoB guidelines:
% 
% At the top of the title page, within the margins, the dissertation should give the title and, if necessary, sub-title and volume number.
% If the dissertation is in a language other than English, the title must be given in that language and in English.
% The full name of the author should be in the centre of the page. At the bottom centre should be the words "A dissertation submitted to the University of Bristol in accordance with the requirements for award of the degree of <degree> in the Faculty of <faculty>", with the name of the school, and month and year of submission.
% The word count of the dissertation (text only) should be entered at the bottom right-hand side of the page.
%
%
\begin{titlingpage}
\begin{SingleSpace}
\calccentering{\unitlength} 
\begin{adjustwidth*}{\unitlength}{-\unitlength}
\vspace*{13mm}
\begin{center}
\rule[0.5ex]{\linewidth}{2pt}\vspace*{-\baselineskip}\vspace*{3.2pt}
\rule[0.5ex]{\linewidth}{1pt}\\[\baselineskip]
{\HUGE A Higgs in the Dark}\\[4mm] % title
{\Large \textit{Searches for dark matter with a focus on invisibly decaying Higgs bosons using the full Run-2 dataset of the CMS experiment at the LHC}}\\ % subtitle
\rule[0.5ex]{\linewidth}{1pt}\vspace*{-\baselineskip}\vspace{3.2pt}
\rule[0.5ex]{\linewidth}{2pt}\\
\vspace{6.5mm}
{\large By}\\
\vspace{6.5mm}
{\large\textsc{Eshwen Bhal}}\\  % name
\vspace{11mm}
\includegraphics[scale=0.6]{logos/bristolcrest_colour}\\  % logo/university crest
\vspace{6mm}
{\large School of Physics\\  % school/department
\textsc{University of Bristol}}\\  % university
\vspace{11mm}
\begin{minipage}{10cm}
A dissertation submitted to the University of Bristol in accordance with the requirements for award of the degree of \textsc{Doctor of Philosophy} in the Faculty of Science.
\end{minipage}\\
\vspace{9mm}
{\large\textsc{July 2020}}  % date of submission
\vspace{12mm}
\end{center}
\begin{flushright}
{\small Word count: number in words}  % word count
\end{flushright}
\end{adjustwidth*}
\end{SingleSpace}
\end{titlingpage}

\clearemptydoublepage

%
% File: abstract.tex
% Author: Victor Brena-Medina
% Description: Contains the text for thesis abstract
%
% UoB guidelines:
%
% Each copy must include an abstract or summary of the dissertation in not more than 300 words, on one side of A4, which should be single-spaced in a font size in the range 10 to 12. If the dissertation is in a language other than English, an abstract in that language and an abstract in English must be included.

\chapter*{Abstract}
\begin{SingleSpace}
\initial{D}ark matter is a poorly-understood phenomenon in nature. Though substantial evidence corroborates its existence, only few characteristics have been determined. One goal of the Large Hadron Collider (LHC) at CERN is to produce dark matter in high energy proton-proton collisions, potentially allowing insight into its currently-mysterious origins. Many models have been postulated regarding its nature, two of which are investigated in this thesis: invisible decays of the Higgs boson, and the production of semi-visible jets. The data used is from the LHC Run-2 era and recorded by the CMS experiment, corresponding to an integrated luminosity of 137\fbinv at a centre of mass energy of 13\TeV.

The branching ratio of the Higgs boson to invisible states is predicted to be 0.1\,\% in the standard model of particle physics. Enhancements from a coupling to dark matter may be observable at the LHC. A search is performed in final states comprising jets and missing transverse momentum targeting the \ttH and \VH Higgs boson production modes. With the full Run-2 dataset from CMS, no significant deviation from the standard model is observed. Results are presented as an upper limit on the measured cross section times branching ratio over the standard model Higgs boson cross section at the 95\,\% confidence level. For the \ttH-tagged events, observed and expected limits of 0.56 and 0.50, respectively, are achieved. For the \VH-tagged events, observed and expected limits of 0.32 and 0.22, respectively, are found. A combined Run-2 limit of 0.28 is observed compared to 0.20 expected. These results are interpreted in simplified dark matter scenarios.

Dark matter may exist in a Hidden Valley dark sector connected to the visible universe via a leptophobic mediator. Analogous to QCD, dark quarks may be produced at the LHC, hadronising and decaying into a mixture of visible and invisible particles: a semi-visible jet. The behaviour of simulated signal from $s$- and \tchannel production modes in the CMS detector---and variables discriminating it from background---is explored. The transverse mass of the dijet system was found to be the most effective.
\end{SingleSpace}
\clearpage

\clearemptydoublepage

%
% file: dedication.tex
% author: Eshwen Bhal
% description: Contains the text for thesis dedication
%

\chapter*{Dedication and acknowledgements}
\begin{SingleSpace}
\initial{T}his work is dedicated to my grandfather, Dato' Mahindar Singh Bhal, who was able to begin this journey with me but sadly unable to finish it.

\

There are far too many people and too little space to individually thank everyone who has accompanied me during this PhD, but I'll try my best.

\

Firstly, to my supervisor, Dr. Henning Fl\"{a}cher. Your advice and guidance over the course of this degree has been instrumental to achieving it, as well as encouraging my growth as a researcher.

\

Secondly, to all my colleagues at the University of Bristol. Very little would have been achieved without your help. From scientific discussions in the pub to pub discussions in the physics building, they have all been fruitful either directly or to allow me to detach from the work. Special thanks are in order to Simone and Sam Maddrell-Mander in my cohort, who have put up with my complaining during the stressful times.

\

To my friends from Monmouth that include my best friends in the world, Mike, James, Sneddon, (Mini) Sam, Matt Bristow, and Sean. You've been on this adventure with me since our school days, and had my back the entire time. We've been through the highest, lowest, and most hilarious times together. I can definitively say I would not be the person I am today without you.

\

To my friends from LTA and those based at CERN, especially Matt Heath, Dwayne and Vukasin. You became my second family while I was in Geneva. I'll miss the skiing, trips into the city, games, and general banter.

\

Lorenza Iacobuzio, you get a special mention. While we weren't especially close until recently, you've been my sage, partner in food, and exceptional friend when I've needed it the most.

\end{SingleSpace}
\clearpage

% Thank the following people in order:
% Physics lot from undergrad that I still hang out with
% Pixie
% End with family - mum, dad, Nadia, brothers and sisters. Include, generally, cousins, uncles, aunties, etc. somewhere

\clearemptydoublepage

\input{frontmatter/declaration}
\clearemptydoublepage

\renewcommand{\contentsname}{Table of Contents}
\maxtocdepth{subsection}
\tableofcontents*
\addtocontents{toc}{\par\nobreak \mbox{}\hfill{\bf Page}\par\nobreak}
\clearemptydoublepage

\listoffigures
\addtocontents{lof}{\par\nobreak\textbf{{\scshape Figure} \hfill Page}\par\nobreak}
\clearemptydoublepage

\listoftables
\addtocontents{lot}{\par\nobreak\textbf{{\scshape Table} \hfill Page}\par\nobreak}
\clearemptydoublepage


% The bulk of the document is delegated to these chapter files in subdirectories.
\mainmatter

\import{chapters/introduction/}{introduction.tex}
\clearemptydoublepage

\import{chapters/theory/}{theory.tex}
\clearemptydoublepage

\import{chapters/detector/}{detector.tex}
\clearemptydoublepage

\import{chapters/higgstoinv/}{higgstoinv.tex}
\clearemptydoublepage

\import{chapters/svj/}{svj.tex}
\clearemptydoublepage

\import{chapters/conclusions/}{conclusions.tex}
\clearemptydoublepage


% And the appendix goes here
\appendix
\import{chapters/appendices/}{app0A.tex}
\clearemptydoublepage


% Apparently the guidelines don't say anything about citations or bibliography styles so I guess we can use anything.
\backmatter

% Bibliography uses bibtex backend
\bibliographystyle{siam} % hopefully fine to use this style. Unsure if references need to be in order they appear in text
\refstepcounter{chapter}
\bibliography{backmatter/thesisbib}
\clearemptydoublepage

% Create separate glossaries for terms and acronyms. Can merge by instead using \printglossaries
\printglossary
\clearpage
\printglossary[type=\acronymtype]
\clearemptydoublepage

% Add index
%\printindex

\end{document}
